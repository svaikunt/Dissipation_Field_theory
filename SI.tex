\documentclass[amsmath,preprintnumbers,10pt,article,notitlepage]{revtex4-1}
\usepackage{amsbsy}
\usepackage{graphicx}
\usepackage{color}
\usepackage{subfigure}
\usepackage{physics}
\usepackage{soul}
\usepackage{color}
\usepackage{bm}
\usepackage[normalem]{ulem}
\usepackage{amsmath}

%\newcommand{\Tr}{\text{Tr}}
\newcommand{\Ai}{\text{Ai}}
\newcommand{\Bi}{\text{Bi}}
\newcommand{\Real}{\text{Re}}
\newcommand{\Imag}{\text{Im}}


\usepackage{verbatim}
\usepackage{natbib}
\bibliographystyle{apsrev4-1}
\begin{document}
\title{Diffusion, structure and energy consumption in a model driven liquid}
\author{Laura Tociu$^{1,3}$, Etienne Fodor$^{2}$, Suriyanarayanan Vaikuntanathan$^{1,3}$} 
\affiliation{$^1$The James Franck Institute, The University of Chicago, Chicago, IL,USA,}
\affiliation{$^2$ DAMTP, University of Cambridge,Cambridge,UK,}
\affiliation{$^3$ Department of Chemistry, The University of Chicago, Chicago, IL, USA}

\maketitle 

\section{Setup and structure of calculations in the SI}


The Hamiltonian for a density field interacting with a tracer can generically be written as:

\begin{equation}
H(t) = \dfrac{K_BT}{2} \int d{\bf r} \int d {\bf r}' \delta \rho ({\bf r}, t) \chi^{-1}({\bf r} - {\bf r}') \delta \rho ({\bf r}', t) + \int d{\bf r} V({\bf r} - {\bf r}_0(t))\delta \rho({\bf r}, t)
\end{equation}
where ${\bf r}_0(t)$ denotes the position of a tracer particle, $\delta \rho({\bf r},t ) = \rho({\bf r}, t) - \rho_0 $ and $\rho_0$ is the density of the liquid. The covariance  is $\chi({\bf r}-{\bf r}')= \langle \delta \rho({\bf r}) \delta \rho ({\bf r}') \rangle=  \rho_0 \delta({\bf r} - {\bf r}') +\rho_0^2 g({\bf r}-{\bf r}')$, and $g({\bf r}-{\bf r}')$ is the two point correlation function. 
\\

Note:  We imagine $\chi^{-1}({\bf r}-{\bf r'})$ to be scaled by the inverse of the square of the integration volume element, so as the double integral above is unitless.
\\

The equation of motion of the density field can then be written as: 
\begin{equation}
\frac{\partial \delta \rho({\bf r}, t)}{\partial t}= - \dfrac{1}{\gamma_G} \nabla^2 \left[ K_BT \int  d{\bf r}' \chi^{-1} ({\bf r}-{\bf r}') \delta \rho({\bf r}', t) +  V({\bf r}-{\bf r}_0(t)) \right ]+ \nabla \cdot \eta({\bf r},t)
\label{eq:EvolutionField}
\end{equation}
The noise is multiplicative, obeying $\langle \eta({\bf r}, t) \cdot \eta({\bf r}', t') \rangle = 2D_G \delta({\bf r} - {\bf r}')\delta(t-t')$, and the parameters $\gamma_G$ and $D_G$ are course-grained damping and diffusion coefficients.


Note:  In the above, if we define a timescale $t_0 = r_0^2/D_0$, we get the following units for the two parameters: $\gamma_G \propto \frac{K_BT V t_0}{L^2}$, and $D_G \propto  \frac{\rho_0^2L^2}{t_0}$.  Thus, the fluctuation dissipation theorem would yield $D_G \gamma_G \propto K_BT V \rho_0^2$.

In Fourier space, Equation \ref{eq:EvolutionField} becomes:

\begin{equation}
\frac{\partial \delta \rho({\bf q},t)}{\partial t}= -\dfrac{1}{\gamma_G}\left[ K_BT |{\bf q}|^2 \chi^{-1}({\bf q}) \delta \rho({\bf q}) +  |{\bf q}|^2 V({\bf q}) e^{-i {\bf q} \cdot {\bf r}_0(t)} \right] + i{\bf q} \cdot \eta({\bf q},t)
\label{eq:FourierEvolutionField}
\end{equation}
where $\langle \eta({\bf q}, s) \cdot \eta({\bf q}', s')  \rangle = 2D_G \delta({\bf q} + {\bf q}') \delta(s-s')$.

Note:  If the interparticle potential were zero, Eq. \ref{eq:FourierEvolutionField} could be solved yielding a probability distribution $P(\delta \rho({\bf q})) = e^{-\tfrac{K_BT \chi^{-1}({\bf q}) \delta \rho({\bf q})^2}{D_G \gamma_G}}$, which would have the right dimensions in the exponent, as expected.

The complementary equation of motion for an undriven tracer is:

\begin{equation}
\dfrac{d{\bf r}_0(t)}{d t} = -\dfrac{1}{\gamma} \int  d{\bf q} i {\bf q} V(-{\bf q})e^{i{\bf q} \cdot {\bf r}_0(t)} \delta \rho({\bf q}, t) +  \eta_0(t)
\label{eq:EvolutionTracer}
\end{equation}
where $\eta_0(t)$ is white noise of zero mean and variance $2D_0$, and $2D_0 = \frac{K_BT}{\gamma}$.
Eq.~\ref{eq:FourierEvolutionField} can be exactly solved if the tracer trajectory, ${\bf r}_0(t)$, is known. Specifically, by labeling $\chi^{-1}({\bf q}) = K({\bf q})$, the restoring force for density fluctuations, we get:

\begin{equation}
\delta\rho({\bf q},t)=
\int\limits_{-\infty}^t e^{-\frac{K_BT}{\gamma_G}|{\bf q}|^2 K({\bf q})(t-s)}\left[-|{\bf q}|^2 \dfrac{V({\bf q})}{\gamma_G} e^{-i {\bf q}\cdot {\bf r}_0(s)}+  i{\bf q} \cdot \eta({\bf q},s)\right] ds
\label{eq:SolutionFourierEvolution}
\end{equation}

As in the main text, we will consider two driven regimes. First, we will consider regimes in which the tracer particle is driven along long orbits with driving frequency $\omega \rightarrow 0$. The derivations presented below explicitly treat that limit. We will then consider the other limit of $\omega \rightarrow \infty$. The expressions for average force, time correlations in the $\omega \rightarrow 0$ can readily be specialized to the case of $\omega \rightarrow \infty$. 

\section{Energy Balance ($\omega \rightarrow 0$)}

We scale the interaction potential between particles by a parameter $h$, and write the full conservative force acting on the tracer as:

\begin{equation}
{\bf F}(t) = -\int d{\bf q} i{\bf q} h V(-{\bf q}) e^{i{\bf q} \cdot {\bf r}_0(t)} \int\limits_{-\infty}^t e^{-\frac{K_BT}{\gamma_G} |{\bf q}|^2 K({\bf q})(t-s)} \left( - |{\bf q}|^2 \dfrac{h V({\bf q})}{\gamma_G} e^{-i{\bf q} \cdot {\bf r}_0(s)} +  i{\bf q} \cdot \eta({\bf q}, s)\right)ds
\label{eq:ConservativeForce}
\end{equation}

We begin by verifying the fluctuation-dissipation theorem at equilibrium. Namely we will prove that $\langle {\bf F}^2 \rangle = \langle {\bf F}^2 \rangle_0 $, where $\displaystyle {\langle {\bf F}^2 \rangle_0 = - \gamma/2 \lim_{t\to 0} \langle \eta_0(0) \cdot {\bf F}(t) \rangle}$.

To order $h^2$, the force variance is:

\begin{align}
\langle {\bf F}^2 \rangle & =  h^2 \left \langle \int d{\bf q} \int  d{\bf q}' i{\bf q}  V(-{\bf q}) i{\bf q}' V({\bf q}') e^{i({\bf q} +{\bf q}') \cdot {\bf r}_0(t)} \int\limits_{-\infty}^t ds \int\limits_{-\infty}^t ds' e^{-\frac{K_BT}{\gamma_G}\left(|{\bf q}|^2 K({\bf q})(t-s)+|{\bf q}'|^2 K({\bf q}')(t-s')\right)} i{\bf q} \eta({\bf q}, s) \cdot  i{\bf q}' \eta({\bf q}', s')  \right \rangle \\
& = h^2 \int d{\bf q} |{\bf q}|^4  |V({\bf q})|^2 2D_G \int\limits_{-\infty}^t ds e^{-2\frac{K_BT}{\gamma_G} |{\bf q}|^2 K({\bf q})(t-s)}  \\
& = h^2 \int d{\bf q} |{\bf q}|^2 \frac{ |V({\bf q})|^2}{K({\bf q})} 
\label{eq:ForceVarianceEquil}
\end{align}

Also to order $h^2$, there are two terms in $\langle {\bf F}^2 \rangle_0 = -\gamma/2  \displaystyle  \lim_{\Delta t \to 0} \langle \eta_0(t) \cdot {\bf F}(t+\Delta t) \rangle$. The first term, which we call $T_1$ is:

\begin{align}
T_1 & =  -\gamma/2   \lim_{\Delta t \to 0}  \left \langle \int d{\bf q}  i{\bf q} \cdot \eta_0(t)  hV(-{\bf q})  e^{i{\bf q} \cdot \int\limits_{-\infty}^{t+\Delta t} \eta_0(x)dx} \int\limits_{-\infty}^{t+\Delta t} e^{-\frac{K_BT}{\gamma_G}|{\bf q}|^2 K({\bf q})(t-s)} |{\bf q}|^2 \dfrac{h  V({\bf q})}{\gamma_G} e^{-i{\bf q} \cdot \int\limits_{-\infty}^{s} \eta_0(x) dx}  ds \right \rangle \\
& =  -\frac{h^2 \gamma}{2\gamma_G}   \lim_{\Delta t \to 0} \int d{\bf q}  |{\bf q}|^2 |V({\bf q})|^2   \int\limits_{-\infty}^{t+\Delta t} e^{-\frac{K_BT}{\gamma_G}|{\bf q}|^2 K({\bf q})(t-s)} \left \langle i{\bf q} \cdot \eta_0(t) e^{i{\bf q} \cdot  \eta_0(t) \Delta t} \right \rangle \left \langle e^{i{\bf q} \cdot \int\limits_{s}^{t} \eta_0(x) dx} \right \rangle ds  \\
& =  \frac{h^2 \gamma}{2\gamma_G}  \int d{\bf q}  |{\bf q}|^4 |V({\bf q})|^2 2D_0  \int\limits_{-\infty}^{t} e^{-\frac{K_BT}{\gamma_G}|{\bf q}|^2 K({\bf q})(t-s) - |{\bf q}|^2 D_0(t-s)} ds  \\
& =  \frac{h^2 K_BT}{\gamma_G} \int d{\bf q} |{\bf q}|^2  \dfrac{ |V({\bf q})|^2}{ \frac{K_BT}{\gamma_G} K({\bf q}) + D_0}   \\
\label{eq:NoiseForceT1Equil}
\end{align}

The second term, $T_2$:

\begin{align}
T_2 & =  \gamma/2   \lim_{\Delta t \to 0}  \left \langle  \int d{\bf q}  i{\bf q} \cdot \eta_0(t)   h V(-{\bf q}) e^{i{\bf q} \cdot \int\limits_{-\infty}^{t+\Delta t} \eta_0(x)dx}  i{\bf q}  \cdot  1/\gamma \int\limits_{-\infty}^{t+\Delta t} ds' (-1) \int d{\bf q}' i{\bf q'}  h V(-{ \bf q}') \right. \\
& \left. e^{i{\bf q}' \cdot \int\limits_{-\infty}^{s'} \eta_0(x)dx} \int\limits_{-\infty}^{s'} e^{-\frac{K_BT}{\gamma_G} |{\bf q}'|^2 K({\bf q'})(t-s'')} i{\bf q}' \cdot \eta({\bf q'}, s'') ds''  \int\limits_{-\infty}^{t+\Delta t} e^{-\frac{K_BT}{\gamma_G}|{\bf q}|^2 K({\bf q})(t-s)} i{\bf q} \cdot \eta({\bf q}, s) ds\right \rangle \\
& = -\frac{h^2}{2}   \lim_{\Delta t \to 0} \int d{\bf q}   |V({\bf q})|^2  |{\bf q}|^4 2D_G \int\limits_{-\infty}^{t+\Delta t} ds'  \left \langle  i{\bf q} \cdot \eta_0(t) e^{i{\bf q} \cdot \int\limits_{s'}^{t+\Delta t} \eta_0(x)dx} \right \rangle \int\limits_{-\infty}^{s'} e^{-2\frac{K_BT}{\gamma_G} |{\bf q}|^2 K({\bf q})(t-s'')} ds'' \\
& = \frac{h^2}{2}   \lim_{\Delta t \to 0} \int d{\bf q}  |V({\bf q})|^2 |{\bf q}|^6 \dfrac{2D_0}{|{\bf q}|^2 K({\bf q})}  \int\limits_{-\infty}^{t} ds' e^{-\frac{K_BT}{\gamma_G} |{\bf q}|^2 K({\bf q})(t-s') - |{\bf q}|^2 D_0(t-s')} \\
& = h^2  \int d{\bf q} |{\bf q}|^2  \dfrac{|V({\bf q})|^2 D_0}{K({\bf q})\left[\frac{K_BT}{\gamma_G} K({\bf q}) + D_0\right]}    \\
\label{eq:NoiseForceT2Equil}
\end{align}

Adding together Eq. \ref{eq:NoiseForceT2Equil} and Eq. \ref{eq:NoiseForceT1Equil} yields exactly the force variance in Eq. \ref{eq:ForceVarianceEquil}:


\begin{align}
\langle {\bf F}^2 \rangle_0 & = T_1 + T_2 \\
& = h^2 \left( \frac{K_BT}{\gamma_G} \int d{\bf q} |{\bf q}|^2  \dfrac{ |V({\bf q})|^2}{ \frac{K_BT}{\gamma_G} K({\bf q}) + D_0} + \int d{\bf q} |{\bf q}|^2  \dfrac{|V({\bf q})|^2 D_0}{K({\bf q})\left[\frac{K_BT}{\gamma_G} K({\bf q}) + D_0\right]} \right) \\
& = h^2 \int d{\bf q} |{\bf q}|^2 \frac{ |V({\bf q})|^2}{K({\bf q})} \\
& = \langle {\bf F}^2 \rangle
\label{eq:VerifyingFDT}
\end{align}



Next we will check what happens when the system is driven out of equilibrium by applying a constant driving force ${\bf v}$ to the tracer. We imagine scaling the interparticle potential by $h$, as before, and further sclaing the driving force by a parameter $Pe$, such that:

\begin{equation}
\dfrac{d {\bf r}_0(t)}{dt} = \dfrac{1}{\gamma}( {\bf F}(t) + Pe {\bf v}) + \eta_0(t)
\end{equation}
where ${\bf F}$ is the force in Eq. \ref{eq:ConservativeForce}, and has already been scaled by the parameter $h$ in the interactions.


In our previous work, energy balance in the system resulted in a relationship between the renormalization of the force fluctuations, $\langle {\bf F}^2 \rangle - \langle {\bf F}^2 \rangle_0$ and the rate of work, defined as $ \langle \dot{w} \rangle = -{\bf v} \cdot \langle {\bf F} \rangle$. This relationship holds as an average over all particles in the system. In the case of a tracer interacting with a fluctuating field, we expect a similar relationship to hold if we consider the rate of energy change of the entire system, tracer and bath together.

The rate of energy change in the tracer is:

\begin{align}
\dfrac{d U}{d t} & = -\langle {\bf F} \cdot \dot {\bf r}_0 \rangle \\
& = -\left( \dfrac{1}{\gamma} \left( \langle {\bf F} ^2 \rangle - \langle {\bf F} ^2 \rangle_0 \right) - \langle \dot{w} \rangle \right) 
\end{align}


Using similar steps as before, we get that to order $h^2$, $\langle {\bf F}^2 \rangle$ is the same as that in Eq. \ref{eq:ForceVarianceEquil}.


The two terms in $\langle {\bf F}^2 \rangle_0 = -\gamma/2   \displaystyle \lim_{\Delta t \to 0} \langle \eta_0(t) \cdot {\bf F}(t+\Delta t) \rangle$, however, are different:

\begin{align}
T_1' & =  \gamma/2   \lim_{\Delta t \to 0}  \left \langle \int d{\bf q}  i{\bf q} \cdot \eta_0(t)  hV({- \bf q})  e^{i{\bf q} \cdot \int\limits_{-\infty}^{t+\Delta t} [ \eta_0(x) + \frac{Pe{\bf v}}{\gamma}] dx} \int\limits_{-\infty}^{t+\Delta t} e^{-\frac{K_BT}{\gamma_G}|{\bf q}|^2 K({\bf q})(t-s)} (-1) |{\bf q}|^2 \dfrac{h  V({\bf q})}{\gamma_G} e^{-i{\bf q} \cdot \int\limits_{-\infty}^{s} [ \eta_0(x) + \frac{Pe{\bf v}}{\gamma} ] dx}  ds \right \rangle \\
& =  -\frac{h^2 \gamma}{2\gamma_G}   \lim_{\Delta t \to 0} \int d{\bf q}  |{\bf q}|^2 |V({\bf q})|^2   \int\limits_{-\infty}^{t+\Delta t} e^{-\frac{K_BT}{\gamma_G}|{\bf q}|^2 K({\bf q})(t-s)} \left \langle i{\bf q} \cdot \eta_0(t) e^{i{\bf q} \cdot  \eta_0(t) \Delta t} \right \rangle \left \langle e^{i{\bf q} \cdot \int\limits_{s}^{t} [ \eta_0(x) + \frac{Pe{\bf v}}{\gamma} ] dx} \right \rangle ds  \\
& =   \frac{h^2 \gamma}{2\gamma_G}  \int d{\bf q}  |{\bf q}|^4 |V({\bf q})|^2 2D_0  \int\limits_{-\infty}^{t} e^{-\frac{K_BT}{\gamma_G}|{\bf q}|^2 K({\bf q})(t-s) - |{\bf q}|^2 D_0(t-s) + i{\bf q} \cdot \frac{Pe{\bf v}}{\gamma}(t-s)} ds  \\
& = \frac{h^2 K_BT}{\gamma_G} \int d{\bf q} |{\bf q}|^4  \dfrac{ |V({\bf q})|^2}{ |{\bf q}|^2 \left[ \frac{K_BT}{\gamma_G} K({\bf q}) + D_0 \right] - i{\bf q} \cdot \frac{Pe{\bf v}}{\gamma}} \\
& = \frac{h^2 K_BT}{\gamma_G} \int d{\bf q} |{\bf q}|^2  \dfrac{ |V({\bf q})|^2}{ \frac{K_BT}{\gamma_G} K({\bf q}) + D_0} \left( 1- \dfrac{ \frac{Pe^2 |{\bf v}|^2}{\gamma^2}}{|{\bf q}|^2 [\frac{K_BT}{\gamma_G} K({\bf q}) + D_0 ]^2 } \right) \\
\label{eq:NoiseForceT1Driven}
\end{align}

The second term, $T_2'$:

\begin{align}
T_2' & =  \gamma/2   \lim_{\Delta t \to 0}  \left \langle \int d{\bf q}  i{\bf q} \cdot \eta_0(t)   h V(-{\bf q}) e^{i{\bf q} \cdot \int\limits_{-\infty}^{t+\Delta t}  [ \eta_0(x) + \frac{Pe{\bf v}}{\gamma}] dx}  i{\bf q}  \cdot  1/\gamma \int\limits_{-\infty}^{t+\Delta t} ds' \int d{\bf q}' i{\bf q'}  h V({ \bf q}') \right. \\
& \left. e^{i{\bf q}' \cdot \int\limits_{-\infty}^{s'} \eta_0(x)dx} \int\limits_{-\infty}^{s'} e^{-\frac{K_BT}{\gamma_G} |{\bf q}'|^2 K({\bf q'})(t-s'')} i{\bf q}' \cdot \eta({\bf q'}, s'') ds''  \int\limits_{-\infty}^{t+\Delta t} e^{-\frac{K_BT}{\gamma_G}|{\bf q}|^2 K({\bf q})(t-s)} i{\bf q} \cdot \eta({\bf q}, s) ds\right \rangle \\
& = -\frac{h^2}{2}   \lim_{\Delta t \to 0} \int d{\bf q}   V^2({\bf q})  |{\bf q}|^4 2D_G \int\limits_{-\infty}^{t+\Delta t} ds'  \left \langle  i{\bf q} \cdot \eta_0(t) e^{i{\bf q} \cdot \int\limits_{s'}^{t+\Delta t}  [ \eta_0(x) + \frac{Pe{\bf v}}{\gamma}] dx} \right \rangle \int\limits_{-\infty}^{s'} e^{-2\frac{K_BT}{\gamma_G} |{\bf q}|^2 K({\bf q})(t-s'')} ds'' \\
& = \frac{h^2}{2}   \lim_{\Delta t \to 0} \int d{\bf q}  |V({\bf q})|^2 |{\bf q}|^6 \dfrac{2D_0}{|{\bf q}|^2 K({\bf q})}  \int\limits_{-\infty}^{t} ds' e^{-\frac{K_BT}{\gamma_G} |{\bf q}|^2 K({\bf q})(t-s') - |{\bf q}|^2 D_0(t-s) + i{\bf q} \cdot \frac{Pe{\bf v}}{\gamma}(t-s)} \\
& = h^2 \int d{\bf q} |{\bf q}|^4  \dfrac{|V({\bf q})|^2 D_0}{K({\bf q})\left[ |{\bf q}|^2 \left[ \frac{K_BT}{\gamma_G} K({\bf q}) + D_0\right] - i {\bf q} \cdot \frac{Pe{\bf v}}{\gamma} \right] }    \\
& = h^2  \int d{\bf q} |{\bf q}|^2  \dfrac{ |V({\bf q})|^2 D_0}{  K({\bf q}) \left[ \frac{K_BT}{\gamma_G} K({\bf q}) + D_0 \right ]} \left( 1 -  \dfrac{ \frac{Pe^2 |{\bf v}|^2}{\gamma^2}}{|{\bf q}|^2 [\frac{K_BT}{\gamma_G} K({\bf q}) + D_0 ]^2 } \right)    \\
\label{eq:NoiseForceT2Driven}
\end{align}

Taken together, the above equations mean that:


\begin{equation}
\langle {\bf F}^2 \rangle_0 = \dfrac{h^2}{\gamma^2} \int d{\bf q} \dfrac{ |{\bf q}|^2 |V({\bf q})|^2}{K({\bf q})} \left( 1 - \dfrac{Pe^2 |{\bf v}|^2}{ |{\bf q}|^2 [\frac{K_BT}{\gamma_G} K({\bf q}) + D_0 ]^2} \right) 
\label{eq:FSqZero}
\end{equation}

and

\begin{equation}
\langle {\bf F}^2 \rangle - \langle {\bf F}^2 \rangle_0 = \dfrac{h^2}{\gamma^2} \int d{\bf q} \dfrac{|V({\bf q})|^2}{K({\bf q})} \dfrac{Pe^2 |{\bf v}|^2}{ [\frac{K_BT}{\gamma_G} K({\bf q}) + D_0 ]^2} 
\label{eq:FDTviolation}
\end{equation}

To calculate the rate of work we need to calculate the average conservative force:

\begin{align}
\langle {\bf F} \rangle & = - \left \langle  \int d{\bf q} i{\bf q} h V({\bf q}) e^{i{\bf q} \cdot {\bf r}_0(t)} \int\limits_{-\infty}^t e^{-\frac{K_BT}{\gamma_G} |{\bf q}|^2 K({\bf q})(t-s)} \left( |{\bf q}|^2 \dfrac{h V({\bf q})}{\gamma_G} e^{-i{\bf q} \cdot {\bf r}_0(s)} +  i{\bf q} \cdot \eta({\bf q}, s)\right)ds  \right \rangle \\
& =   - \dfrac{1}{\gamma_G} \int d{\bf q} i{\bf q} |{\bf q}|^2 h |V({\bf q})|^2 \int\limits_{-\infty}^t e^{-\frac{K_BT}{\gamma_G} |{\bf q}|^2 K({\bf q})(t-s') - |{\bf q}|^2 D_0(t-s) + i{\bf q} \cdot  \frac{Pe{\bf v}}{\gamma}(t-s)} \\
& - \left \langle \int d{\bf q} i{\bf q} h V({\bf q}) e^{i{\bf q} \cdot \int\limits_{-\infty}^{t} ( \eta_0(x) +  \frac{Pe{\bf v}}{\gamma} ) dx} i{\bf q} \cdot \int\limits_{-\infty}^{t} ds' \dfrac{1}{\gamma} \int d{\bf q'} i{\bf q}' h V({\bf q}') e^{i{\bf q}' \cdot \int\limits_{-\infty}^{s'} (\eta_0(x) +  \frac{Pe{\bf v}}{\gamma}) dx} \right. \\ 
& \left. \int\limits_{-\infty}^{s'} ds''  \int\limits_{-\infty}^{t} ds e^{-\frac{K_BT}{\gamma_G}|{\bf q}|^2 K({\bf q})(t-s)}    e^{-\frac{K_BT}{\gamma_G}|{\bf q}|^2 K({\bf q})(t-s)}i{\bf q}' \cdot \eta({\bf q}', s'') i{\bf q} \cdot \eta({\bf q}, s)   \right \rangle \\
& = - \dfrac{h^2}{\gamma_G} \int d{\bf q} |{\bf q}|^2 i{\bf q} \dfrac{|V({\bf q})|^2}{|{\bf q}|^2 \left[\frac{K_BT}{\gamma_G} K({\bf q}) + D_0 \right] - i{\bf q} \cdot  \frac{Pe{\bf v}}{\gamma}} \\
& - \dfrac{h^2}{\gamma}\int d{\bf q} |{\bf q}|^4 i{\bf q} |V({\bf q})|^2 \left \langle \int\limits_{-\infty}^{t} ds' e^{i{\bf q}\int_s^t (\eta_0(x) +  \frac{Pe{\bf v}}{\gamma}) dx} \right \rangle 2D_G \int\limits_{-\infty}^{s'} ds''  e^{-\frac{K_BT}{\gamma_G} 2K({\bf q})(t-s'') }\\
& = - \dfrac{h^2}{\gamma}  \int d{\bf q} |{\bf q}|^2  i{\bf q} \dfrac{|V({\bf q})|^2|}{|{\bf q}|^2 \left[\frac{K_BT}{\gamma_G} K({\bf q}) + D_0 \right] - i{\bf q} \cdot  \frac{Pe{\bf v}}{\gamma}} \left( \dfrac{\gamma}{\gamma_G} + \dfrac{1}{K({\bf q})} \right) \\
& = - \dfrac{h^2}{\gamma D_0}  \int d{\bf q} |{\bf q}|^2  i{\bf q} \dfrac{|V({\bf q})|^2}{|{\bf q}|^2 \left[\frac{K_BT}{\gamma_G} K({\bf q}) + D_0 \right] - i{\bf q} \cdot  \frac{Pe{\bf v}}{\gamma}} \left( \dfrac{K_BT}{\gamma_G} + \dfrac{D_0}{K({\bf q})} \right) \\
& = - \dfrac{h^2}{K_BT} \int d{\bf q} |{\bf q}|^2 i{\bf q} \dfrac{|V({\bf q})|^2 \left[\frac{K_BT}{\gamma_G} K({\bf q}) + D_0 \right]}{K({\bf q}) \left[ |{\bf q}|^2 \left( \frac{K_BT}{\gamma_G} K({\bf q}) + D_0 \right) - i{\bf q} \cdot  \frac{Pe{\bf v}}{\gamma} \right]} \\
& =  - \dfrac{h^2}{K_BT \gamma} \int d{\bf q} \dfrac{|V({\bf q})|^2 Pe {\bf v}}{K({\bf q}) \left[ \frac{K_BT}{\gamma_G} K({\bf q}) + D_0 \right]} \\
\end{align}

The rate of work done on the system, in this case, is:

\begin{equation}
\langle \dot{w} \rangle = -\dfrac{1}{\gamma} Pe {\bf v} \cdot \langle F \rangle =  \dfrac{h^2 Pe^2}{K_BT \gamma^2} \int d{\bf q} \dfrac{|V({\bf q})|^2  |{\bf v}|^2}{K({\bf q}) \left[ \frac{K_BT}{\gamma_G} K({\bf q}) + D_0 \right]}
\label{eq:RateOfWork}
\end{equation}

All in all, the change in energy of the tracer becomes:

\begin{align}
\dfrac{dU}{dt} & = -\left( \dfrac{1}{\gamma} \left( \langle {\bf F} ^2 \rangle - \langle {\bf F} ^2 \rangle_0 \right) - \langle \dot{w} \rangle \right) \\
& = -\dfrac{h^2Pe^2 |\bf v|^2}{\gamma^2 K_BT} \int d{\bf q} \left( \dfrac{K_BT}{\gamma} \dfrac{|V({\bf q})|^2}{K({\bf q}) \left[\frac{K_BT}{\gamma_G} K({\bf q}) + D_0 \right]^2} - \dfrac{|V({\bf q})|^2}{K({\bf q}) \left[\frac{K_BT}{\gamma_G} K({\bf q}) + D_0 \right]}  \right) \\
& =  \dfrac{h^2 Pe^2 |\bf v|^2}{\gamma^2 \gamma_G} \int d{\bf q} \dfrac{|V({\bf q})|^2}{\left[\frac{K_BT}{\gamma_G} K({\bf q}) + D_0 \right]^2}  
\label{eq:dutracer}
\end{align}


\begin{align}
\left \langle \dfrac{dU_{\mbox{bath}}}{dt}  \right \rangle & = \left \langle \int d {\bf q} |{\bf q}|^2 h V(-{\bf q}) e^{i{\bf q} \cdot {\bf r}_0(t)} \delta \dot{\rho}({\bf q}) \right \rangle \\
& = - \left \langle \int d {\bf q} h V(-{\bf q}) e^{i{\bf q} \cdot {\bf r}_0(t)} \left( \dfrac{1}{\gamma_G}\left[ K_BT |{\bf q}|^2 K({\bf q}) \delta \rho({\bf q}) +  |{\bf q}|^2 h V({\bf q}) e^{-i {\bf q} \cdot {\bf r}_0(t)} \right] + i{\bf q} \cdot \eta({\bf q},t) \right)  \right \rangle
\end{align}

We get a total of two terms, which doing similar manipulations as before, to order $h^2$ end up being:

\begin{align}
U_1 &  = - \left \langle \int d {\bf q} h V(-{\bf q}) e^{i{\bf q} \cdot {\bf r}_0(t)} \dfrac{K_BT}{\gamma_G} |{\bf q}|^2 K({\bf q}) \delta \rho({\bf q})   \right \rangle \\
& =  - \left \langle \int d {\bf q} h V(-{\bf q}) e^{i{\bf q} \cdot {\bf r}_0(t)} \dfrac{K_BT}{\gamma_G} |{\bf q}|^2 K({\bf q} )    \int\limits_{-\infty}^t e^{-\frac{K_BT}{\gamma_G}|{\bf q}|^2 K({\bf q})(t-s)}\left[-|{\bf q}|^2 \dfrac{h V({\bf q})}{\gamma_G} e^{-i {\bf q}\cdot {\bf r}_0(s)}+  i{\bf q} \cdot \eta({\bf q},s)\right] ds   \right \rangle  \\
& =  \int d {\bf q} h V(-{\bf q})  \dfrac{K_BT}{\gamma_G} |{\bf q}|^2 K({\bf q} )    \int\limits_{-\infty}^t e^{-\frac{K_BT}{\gamma_G}|{\bf q}|^2 K({\bf q})(t-s)} |{\bf q}|^2 \dfrac{h V({\bf q})}{\gamma_G}  \left \langle  e^{i{\bf q} \cdot \int\limits_{s}^t ( \eta_0(x) + \frac{Pe{\bf v} }{\gamma}) dx}   \right \rangle ds   - \\
& \left \langle \int d {\bf q} h V(-{\bf q}) e^{i{\bf q} \cdot {\bf r}_0(t)} \dfrac{K_BT}{\gamma_G} |{\bf q}|^2 K({\bf q} )    \int\limits_{-\infty}^t e^{-\frac{K_BT}{\gamma_G}|{\bf q}|^2 K({\bf q})(t-s)} i{\bf q} \cdot \eta({\bf q},s) ds   \right \rangle \\
& = \frac{h^2}{\gamma_G} \int d{\bf q} |{\bf q}|^2  \dfrac{ |V({\bf q})|^2}{ \frac{K_BT}{\gamma_G} K({\bf q}) + D_0} \left( 1- \dfrac{ \frac{Pe^2 |{\bf v}|^2}{\gamma^2}}{|{\bf q}|^2 [\frac{K_BT}{\gamma_G} K({\bf q}) + D_0 ]^2 }  \right) \left( \dfrac{K_B T}{\gamma_G}K({\bf q}) + D_0 \right) \\
& = \frac{h^2}{\gamma_G} \int d{\bf q} |{\bf q}|^2   |V({\bf q})|^2\left( 1- \dfrac{ \frac{Pe^2 |{\bf v}|^2}{\gamma^2}}{|{\bf q}|^2 [\frac{K_BT}{\gamma_G} K({\bf q}) + D_0 ]^2 }  \right)
\end{align}

The second term is very simply:

\begin{align}
U_2 &  = -\frac{h^2}{\gamma_G} \int d{\bf q} |{\bf q}|^2   |V({\bf q})|^2 
\end{align}


Together, the rate of change of the energy of the bath becomes:

\begin{align}
\left \langle \dfrac{dU_{\mbox{bath}}}{dt}  \right \rangle & = U_1 + U_2 \\
& = - \frac{h^2 Pe^2 |{\bf v}|^2}{ \gamma^2 \gamma_G} \int d{\bf q}  \dfrac{|V({\bf q})|^2}{ [\frac{K_BT}{\gamma_G} K({\bf q}) + D_0 ]^2 } 
\label{eq:dubath}
\end{align}

Since Eq. \ref{eq:dubath} is the negative of Eq. \ref{eq:dutracer}, this completes the proof that the energy of the system remains constant.

Since $K({\bf q}) = \rho^{-2} $ to zeroth order in $h$, we obtain that $\dfrac{KT}{\gamma_G} K({\bf q}) \equiv D_0$, and thus Eq. \ref{eq:FDTviolation} and Eq. \ref{eq:RateOfWork} show that $\frac{2}{\gamma}(\langle {\bf F}^2 \rangle - \langle {\bf F}^2 \rangle_0) = \langle \dot{w} \rangle$.

\section{Work Renormalizes Diffusion Coefficient ($\omega \rightarrow 0$) }

We will now calculate the correction to the equilibrium diffusion coefficient of the driven tracer. 
In overdamped Langevin dynamics, the expression for diffusion in terms of force-force and noise-force correlations is: 
\begin{equation}
D = D_0 + \frac{1}{d} \int_0^\infty  \left[\left \langle \frac{({\bf F}(0) - \langle {\bf F} \rangle) \cdot ({\bf F}(t)-\langle {\bf F} \rangle)}{\gamma^2}\right \rangle  + \left \langle \frac{{\bm \eta}(0) \cdot {\bf F}(t)}{\gamma} \right \rangle \right]{d}t 
\label{eq:diffusion}
\end{equation}

However, the terms involving the average conservative force lead to terms of order $h^4$, so we can ignore them.

To proceed with the calculation of the nonequilibrium diffusion coefficient, we will first evaluate the first integral:


\begin{align}
\int_0^\infty  \left \langle \dfrac{ {\bf F}(0) \cdot {\bf F}(t)}{\gamma^2} \right \rangle dt  & = \dfrac{h^2}{\gamma^2}  \int_0^\infty dt \left \langle \int d{\bf q} \int  d{\bf q}' i{\bf q}  V(-{\bf q}) i{\bf q}' V({-\bf q}') e^{i{\bf q} \cdot {\bf r}_0(0)  + i {\bf q}' \cdot {\bf r}_0(t)} \right . \\ & \left . \int\limits_{-\infty}^0 ds \int\limits_{-\infty}^t ds' e^{-\frac{K_BT}{\gamma_G}\left(|{\bf q}|^2 K({\bf q})(-s)+|{\bf q}'|^2 K({\bf q}')(t-s')\right)} i{\bf q} \eta({\bf q}, s) \cdot  i{\bf q}' \eta({\bf q}', s')  \right \rangle \\
& = \dfrac{h^2}{\gamma^2}  \int_0^\infty dt \int d{\bf q} |{\bf q}|^4  |V({\bf q})|^2 2D_G e^{-\frac{K_BT}{\gamma_G} |{\bf q}|^2 K({\bf q})t} e^{-|{\bf q}|^2 D_0 t - i{\bf q} \cdot \frac{Pe {\bf v}}{\gamma} t} \int\limits_{-\infty}^0 e^{2\frac{K_BT}{\gamma_G} |{\bf q}|^2 K({\bf q})s}ds \\
& = \dfrac{h^2}{\gamma^2}  \int d{\bf q}  \dfrac{ |{\bf q}|^2 |V({\bf q})|^2 } {K({\bf q}) \left( \frac{K_BT}{\gamma_G} |{\bf q}|^2 K({\bf q}) + |{\bf q}|^2D_0 + i{\bf q} \cdot \frac{Pe {\bf v}}{\gamma} \right)}   \\
& = \dfrac{h^2}{\gamma^2}  \int d{\bf q} \dfrac{ |{\bf q}|^2 |V({\bf q})|^2 \left ( \frac{K_BT}{\gamma_G} |{\bf q}|^2 K({\bf q}) + |{\bf q}|^2D_0 \right )} { K({\bf q}) \left( ( \frac{K_BT}{\gamma_G} |{\bf q}|^2 K({\bf q}) +  |{\bf q}|^2 D_0 )^2 + |{\bf q}|^2 \frac{Pe^2 |v|^2}{\gamma^2} \right)} \\
& =  \dfrac{h^2}{\gamma^2}  \int d{\bf q} \dfrac{ |V({\bf q})|^2 } { K({\bf q}) ( \frac{K_BT}{\gamma_G}  K({\bf q}) +  D_0 )} \left( 1- \dfrac{|{\bf q}|^2 \frac{Pe^2 |v|^2}{\gamma^2}}{\left(  \frac{K_BT}{\gamma_G} |{\bf q}|^2 K({\bf q}) +  |{\bf q}|^2 D_0 \right)^2} \right)
\end{align}

The second term in Eq. \ref{eq:diffusion} has, in turn, two terms: 


\begin{align}
T_1'' & =  - \dfrac{1}{\gamma} \int_0^{\infty} dt \left \langle \int d{\bf q}  i{\bf q} \cdot \eta_0(0)  hV({- \bf q})  e^{i{\bf q} \cdot \int\limits_{-\infty}^{t} [ \eta_0(x) + \frac{Pe{\bf v}}{\gamma}] dx} \int\limits_{-\infty}^{t} e^{-\frac{K_BT}{\gamma_G}|{\bf q}|^2 K({\bf q})(t-s)} (-1) |{\bf q}|^2 \dfrac{h  V({\bf q})}{\gamma_G} e^{-i{\bf q} \cdot \int\limits_{-\infty}^{s} [ \eta_0(x) + \frac{Pe{\bf v}}{\gamma} ] dx}  ds \right \rangle \\
& =   \frac{h^2}{\gamma_G \gamma}  \int_0^{\infty}  dt  \int d{\bf q}  |{\bf q}|^2 |V({\bf q})|^2   \int\limits_{-\infty}^{0} e^{-\frac{K_BT}{\gamma_G}|{\bf q}|^2 K({\bf q})(t-s)} \left \langle i{\bf q} \cdot \eta_0(0) e^{i{\bf q} \cdot  \eta_0(0)} \right \rangle \left \langle e^{i{\bf q} \cdot \int\limits_{s}^{t} [ \eta_0(x) + \frac{Pe{\bf v}}{\gamma} ] dx} \right \rangle ds  \\
& =  -\frac{h^2}{\gamma_G \gamma}   \int_0^{\infty}  dt \int d{\bf q}  |{\bf q}|^4 |V({\bf q})|^2 2D_0  \int\limits_{-\infty}^{0} e^{-\frac{K_BT}{\gamma_G}|{\bf q}|^2 K({\bf q})(t-s) - |{\bf q}|^2 D_0(t-s) + i{\bf q} \cdot \frac{Pe{\bf v}}{\gamma}(t-s)} ds  \\
& = - 2\frac{h^2 D_0}{\gamma_G \gamma} \int d{\bf q} \dfrac{ |{\bf q}|^4 |V({\bf q})|^2}{ \left(|{\bf q}|^2 \left[ \frac{K_BT}{\gamma_G} K({\bf q}) + D_0 \right] - i{\bf q} \cdot \frac{Pe{\bf v}}{\gamma} \right)^2} \\
\label{eq:NoiseForceDiffusion1}
\end{align}

Labeling $|{\bf q}|^2 \left[ \frac{K_BT}{\gamma_G} K({\bf q}) + D_0 \right] = A$, we can rewrite the above as:

\begin{align}
T_1'' & = - 2\frac{h^2 D_0}{\gamma_G \gamma} \int d{\bf q} \dfrac{ |{\bf q}|^4 |V({\bf q})|^2}{ (A - i{\bf q} \cdot \frac{Pe{\bf v}}{\gamma} )^2} \\
& = - 2 \frac{h^2 D_0}{\gamma_G \gamma} \int d{\bf q}  |{\bf q}|^4 |V({\bf q})|^2 \dfrac{A^2 - |{\bf q}|^2 \frac{Pe^2 |{\bf v}|^2}{\gamma^2}}{\left( A^2 + |{\bf q}|^2 \frac{Pe^2 |{\bf v}|^2}{\gamma^2} \right)^2} \\
& = -2 \frac{h^2 D_0}{\gamma_G \gamma} \int d{\bf q}  \dfrac{ |{\bf q}|^4 |V({\bf q})|^2}{A^2} \left( 1 - 3 |{\bf q}|^2 \frac{Pe^2 |{\bf v}|^2}{\gamma^2 A^2}  \right) \\
\label{eq:NoiseForceDiffusion1cont}
\end{align}


The second term, $T_2''$, we compute next, and we make the same substitution (i.e. $|{\bf q}|^2 \left[ \frac{K_BT}{\gamma_G} K({\bf q}) + D_0 \right] = A$) in last line:

\begin{align}
T_2'' & =  - \dfrac{1}{\gamma}  \int_0^{\infty} dt \left \langle \int d{\bf q}  i{\bf q} \cdot \eta_0(0)   h V(-{\bf q}) e^{i{\bf q} \cdot \int\limits_{-\infty}^{t}  [ \eta_0(x) + \frac{Pe{\bf v}}{\gamma}] dx}  i{\bf q}  \cdot  1/\gamma \int\limits_{-\infty}^{t} ds' \int d{\bf q}' i{\bf q'}  h V({ \bf q}') \right. \\
& \left. e^{i{\bf q}' \cdot \int\limits_{-\infty}^{s'} \eta_0(x)dx} \int\limits_{-\infty}^{s'} e^{-\frac{K_BT}{\gamma_G} |{\bf q}'|^2 K({\bf q'})(t-s'')} i{\bf q}' \cdot \eta({\bf q'}, s'') ds''  \int\limits_{-\infty}^{t} e^{-\frac{K_BT}{\gamma_G}|{\bf q}|^2 K({\bf q})(t-s)} i{\bf q} \cdot \eta({\bf q}, s) ds\right \rangle \\
& = \frac{h^2}{\gamma^2}  \int_0^{\infty} dt  \int d{\bf q}   V^2({\bf q})  |{\bf q}|^4 2D_G \int\limits_{-\infty}^{t} ds'  \left \langle  i{\bf q} \cdot \eta_0(0) e^{i{\bf q} \cdot \int\limits_{s'}^{t}  [ \eta_0(x) + \frac{Pe{\bf v}}{\gamma}] dx} \right \rangle \int\limits_{-\infty}^{s'} e^{-2\frac{K_BT}{\gamma_G} |{\bf q}|^2 K({\bf q})(t-s'')} ds'' \\
& = -\frac{h^2}{\gamma^2}  \int_0^{\infty} dt  \int d{\bf q}  |V({\bf q})|^2 |{\bf q}|^6 \dfrac{2D_0}{|{\bf q}|^2 K({\bf q})}  \int\limits_{-\infty}^{0} ds' e^{-\frac{K_BT}{\gamma_G} |{\bf q}|^2 K({\bf q})(t-s') - |{\bf q}|^2 D_0(t-s) + i{\bf q} \cdot \frac{Pe{\bf v}}{\gamma}(t-s)} \\
& = -2 \dfrac{h^2 D_0}{\gamma^2} \int d{\bf q} |{\bf q}|^4  \dfrac{|V({\bf q})|^2}{K({\bf q})\left[ |{\bf q}|^2 \left[ \frac{K_BT}{\gamma_G} K({\bf q}) + D_0\right] - i {\bf q} \cdot \frac{Pe{\bf v} }{\gamma} \right]^2 } \\
& = -2 \frac{h^2 D_0}{\gamma^2} \int d{\bf q}  \dfrac{|{\bf q}|^4 |V({\bf q})|^2}{K({\bf q})} \dfrac{A^2 - |{\bf q}|^2 \frac{Pe^2 |{\bf v}|^2}{\gamma^2}}{\left( A^2 + |{\bf q}|^2 \frac{Pe^2 |{\bf v}|^2}{\gamma^2} \right)^2} \\
& = -2 \frac{h^2 D_0}{\gamma^2} \int d{\bf q}  \dfrac{ |{\bf q}|^4 |V({\bf q})|^2}{K({\bf q}) A^2} \left( 1 - 3 |{\bf q}|^2 \frac{Pe^2 |{\bf v}|^2}{\gamma^2 A^2}  \right) \\
\label{eq:NoiseForceDiffusion2}
\end{align}

Adding together $T_1''$ and $T_2''$, we obtain:


\begin{equation}
\int_0^\infty  \left \langle \dfrac{ {\bf \eta_0}(0) \cdot {\bf F}(t)}{\gamma} \right \rangle dt = - 2\dfrac{h^2}{\gamma^2}  \int d{\bf q} \dfrac{ |V({\bf q})|^2 } { K({\bf q}) ( \frac{K_BT}{\gamma_G}  K({\bf q}) +  D_0 )} \left( 1- 3 \dfrac{|{\bf q}|^2 \frac{Pe^2 |v|^2}{\gamma^2}}{\left(  \frac{K_BT}{\gamma_G} |{\bf q}|^2 K({\bf q}) +  |{\bf q}|^2 D_0 \right)^2} \right)
\end{equation}


Thus, the full formula for the diffusion coefficient becomes:

\begin{equation}
D = D_0 - \dfrac{h^2}{\gamma^2}  \int d{\bf q} \dfrac{ |V({\bf q})|^2 } { K({\bf q}) ( \frac{K_BT}{\gamma_G}  K({\bf q}) +  D_0 )} + 5 \dfrac{h^2}{\gamma^2}  \int d{\bf q} \dfrac{ |V({\bf q})|^2 } { K({\bf q}) ( \frac{K_BT}{\gamma_G}  K({\bf q}) +  D_0 )}  \dfrac{|{\bf q}|^2 \frac{Pe^2 |v|^2}{\gamma^2}}{\left(  \frac{K_BT}{\gamma_G} |{\bf q}|^2 K({\bf q}) +  |{\bf q}|^2 D_0 \right)^2} 
\end{equation}


\section{Computations in the limit ($\omega \rightarrow \infty$) }
The effects of the driving force show up in the above sections through integrals of the form 
\begin{equation}
I_1(t)= \int_{-\infty}^{t} e^{-G(q) (t-s)} e^{- i {\bf q} \int_s^t {Pe {\bf v}(s') ds'}} ds 
\label{eq:centralIntegral}
\end{equation}
where ${\bf v}(s) =\cos(\omega s) \hat {\bf e}_x + \sin(\omega s) \hat{\bf e}_y$. In the perturbation theory for the average force exerted by the medium on the tracer particle, $G(q) \equiv (K(q) + D_0)q^2 $. Similar expressions for $G(q)$ can be obtained for the other calculations. 

When $\omega\rightarrow 0$, $\int_s^t Pe{\bf v}(s) ds = Pe \hat {\bf e}_x (t-s)$ and we recover the expressions obtained in the previous section. 
In the limit $\omega\gg G(q)$ the integral $I_1$ can be evaluated as (at $t=0$)
\begin{equation}
I_1(0)= \frac{1}{G(q)} + i\left( \frac{q_y Pe}{\omega G(q)} - \frac{q_x Pe}{\omega^2}\right)- \frac{ q_x^2 Pe^2 }{ 4 G(q) \omega^2} -\frac{5 q_y^2 Pe^2}{4 G(q)\omega^2}
\label{eq:centralIntegral2}
\end{equation}

As is clear from the perturbative expansions of the force and correlation functions, the imaginary part of Eq.~\ref{eq:centralIntegral2} contributes to the forces while the real part contributes to the correlation functions. Hence, to leading order, the forces in the limit $\omega \rightarrow \infty$ scale like $Pe/\omega$. Hence the total work performed in a cycle scales as $Pe^2/\omega^2$. Similarly, the renormalization of the diffusion constant scales like $Pe^2/\omega^2$. 


% ....................XXXXXXXX...........................................



% In order to determine how the $Pe^2$ scaling affects the isotropic pair correlation function, we need to express $g_3({\bf r}, {\bf r}')$ in terms of $g({\bf r})$ and $g({\bf r}')$. To do so we use the observation in Fig. 3 of the main text, namely that the renormalization of the pair correlation function, which we call $\delta(r) = g(r) - g_{eq}(r)$, at contact (ie around $\sigma$) seems to be well approximated as a square function of a height that scales like $Pe^2$ and a width $2 \Delta \sigma$ that is on the order of 1. Considering $\Delta \sigma$ to be small, we can approximate integrals of isotropic functions $f({\bf r})$ as (See Fig. \ref{fig:GrAsSquareFunction})  : 
% \begin{align}
% \langle f \rangle & = \int d{\bf r} f({\bf r}) \rho g({\bf r}) \equiv \rho \int_0^{\sigma_{-}} f({\bf r}) \rho g({\bf r}) +  \rho \int_{\sigma_{+}}^{\infty} f({\bf r}) \rho g({\bf r}) \\
% &= \langle f \rangle_{eq} + 2 \rho \Delta \sigma 2\pi \sigma f(\sigma) \delta(\sigma)
% \end{align}

% Hence, we can make the following approximations:

% \begin{equation}
% \langle {\bf F}^2 \rangle_0 = \langle {\bf F}^2 \rangle_0 ({\mbox{equil}}) + 2\Delta \sigma 2\pi \sigma \rho K_B T \grad^2 u(\sigma) \delta(\sigma)
% \end{equation}

% and

% \begin{equation}
% \langle {\bf F}^2 \rangle = \langle {\bf F}^2 \rangle ({\mbox{equil}}) + 2 \Delta \sigma 2\pi \sigma \rho (\grad u(\sigma))^2 \delta(\sigma)  + A
% \end{equation}
% where $A = \rho^2 \int  \int  \grad u({\bf r}) \cdot  \grad u({\bf r}')  g_3({\bf r}, {\bf r}') d{\bf r} d{\bf r}' - \rho^2 \int  \int  \grad u({\bf r}) \cdot  \grad u({\bf r}')  g_3({\bf r}, {\bf r}') d{\bf r} d{\bf r}' ({\mbox{equil}})$.


% Note:  Here we have to explain the assumptions (hard sphere approximation etc)

% We can separate $A$ into three parts:

% \begin{align}
% A  +  \rho^2 \int  \int  \grad u({\bf r}) \cdot  \grad u({\bf r}')  g_3({\bf r}, {\bf r}') d{\bf r} d{\bf r}' ({\mbox{equil}})& =  \rho^2 \int_0^{\sigma_{-}}  \int_0^{\sigma_{-}}  \grad u({\bf r}) \cdot  \grad u({\bf r}')  g_3({\bf r}, {\bf r}') d{\bf r} d{\bf r}' + \\
% & 2 \rho^2 \int_0^{\sigma_{-}}  \int_{\sigma_{+}}^{\infty}   \grad u({\bf r}) \cdot  \grad u({\bf r}')  g_3({\bf r}, {\bf r}') d{\bf r} d{\bf r}' + \\
% & \rho^2 \int_{\sigma_{+}}^{\infty} \int_{\sigma_{+}}^{\infty} \grad u({\bf r}) \cdot  \grad u({\bf r}')  g_3({\bf r}, {\bf r}') d{\bf r} d{\bf r}'
% \end{align}

% The first term is zero for any values of ${\bf r}$ and ${\bf r}'$ that are not exactly $\sigma_{-} - \Delta \sigma$ (basically the integration unit distance preceding $\sigma_{-}$). When both of these distance are exactly that, it means two particles are exactly at a distance $\sigma$, leaving a probability of 2/3 that the center of the third particle, which is also restrained to be on a circle of radius $\equiv \sigma$ around the first particle, does not overlap with the second particle.  The probability of finding such a configuration is therefore $g(\sigma)^2 * 2/3$

% The second term fixes two particles at a distance $ r' >$ $\sigma$, and computes the probability that the third particle is at a distance $\sigma$ from one of the two and does not intersect the other one. This probability is $1-2arccos(\sigma/2r')/2\pi$. However, since we are assuming the simple shape of the pair correlation function in Fig. \ref{fig:GrAsSquareFunction}, the relevant probability will again be $g(\sigma)^2 * 2/3$. 

% Lastly, the third term can be approximated as consisting of independent non-interacting particles, yielding simply $g(\sigma)^2$.

% All in all, we get that:

% \begin{align}
%  A = \rho^2 f (\sigma)^2 \delta(\sigma)^2 f(\sigma)^2 (2\pi \sigma)^2 (\Delta \sigma /2)^2
% \end{align}

% These results show us that $\delta(\sigma)$ should scale like $Pe^2$.

% (To be filled in later:  more details  +  add figure  +  quantitative analysis of how well this approximation holds)

\pagebreak

\section*{References}

\begin{thebibliography}{1}

\bibitem{Kruger2017}
Kruger M, Dean DS (2017) A gaussian theory for fluctuations in simple liquids.
\newblock {\em The Journal of Chemical Physics} 146(13):134507.

\end{thebibliography}

\end{document}
