\documentclass[pre, superscriptaddress, twocolumn,pre]{revtex4-1}
\usepackage{amsmath, amssymb, color, graphicx, stmaryrd, wasysym, esint}
\definecolor{linkcolor}{rgb}{0,0,0.6} 
\usepackage[pdftex,colorlinks=true,
	pdfstartview = FitV,
	linkcolor    = linkcolor,
	citecolor    = linkcolor,
	urlcolor     = linkcolor,	
	hyperindex   = true,
	hyperfigures = false]{hyperref}

\newcommand{\dd}{\text{d}}
\newcommand{\ee}{\text{e}}
\newcommand{\ii}{\text{i}}


% ===============================================================================


\begin{document}

\title{How dissipation constrains fluctuations in driven liquids:\\Diffusion, structure and biased interactions}
\author{Laura Tociu}
\affiliation{James Franck Institute, University of Chicago, Chicago, IL 60637}
\affiliation{Department of Chemistry, University of Chicago, Chicago, IL 60637}

\author{\'Etienne Fodor}
\affiliation{DAMTP, Centre for Mathematical Sciences, University of Cambridge, Wilberforce Road, Cambridge CB3 0WA, UK}

\author{Takahiro Nemoto}
\affiliation{Philippe Meyer Institute for Theoretical Physics, Physics Department, \'Ecole Normale Sup\'erieure \& PSL Research University, 24, rue Lhomond, 75231 Paris Cedex 05, France}

\author{Suriyanarayanan Vaikuntanathan}
\affiliation{James Franck Institute, University of Chicago, Chicago, IL 60637}
\affiliation{Department of Chemistry, University of Chicago, Chicago, IL 60637}

\begin{abstract}

The dynamics and structure of driven liquids generically hold the signature of a net dissipation of energy in the thermostat. Yet, disentangling precisely how dissipation changes collective effects remains challenging in many-body systems due to the complex interplay between driving and interactions. Combining explicit coarse-graining and stochastic calculus, we obtain simple relations between tracer diffusion, density correlations and dissipation. Importantly, this suggests that tuning dissipation provides a route to alter liquid structure in a controlled manner. It has been proposed that dissipation can be monitored with a dynamics bias, we rather consider here large-deviation biased ensembles where trajectories mimic the effect of an external drive. We put forward some specific bias which leads to renormalize interactions and to promote spontaneous clustering. By biasing targeted pairs of particles, atypical structures can be stabilized in a controlled fashion. This is illustrated by numerical simulations of a minimal case study, where sampling is biased with a cloning algorithm. Finally, we argue that such biased trajectories provide some intuition on how non-equilibrium forces can lead to spatial organization, with potential relevance for complex self-assembly.

\end{abstract}

\maketitle 


% ===============================================================================


\section{Introduction}

Nonequilibrium forces can drive novel and specific pathways to modulate phase transitions and self-assembly in materials. The close connection between the net dissipation of energy, powered by these forces, internal transport and spatial organization is especially apparent in living systems~\cite{Toyabe2010, Ahmed2016, Battle604, Mura2018}. As an example, the flagella motors of {\it E. Coli} exhibit a unique phenomenology combining ultra-sensitive response, adaptation, and motor restructuring as a function of the applied load~\cite{Lele2013, Lan2012, Wang2017}. Moreover, {\it in vivo} studies of the cellular cytoskeleton, as well as {\it in vitro} experiments on reconstituted systems, have also shown that motor-induced forces control a large variety of functionality in the cell~\cite{Silva2011, Sanchez2012, Blanchoin2014, Murrell2015, Decamp2015}.


To elucidate the role of nonequilibrium forces in materials, it is crucial to examine how dissipation affects the emerging dynamics and structure. While equilibrium features are well established, progress in controlling systems with sustained dissipation has been hampered by a lack of general principles~\cite{Cates2015, Solon2015a, Nguyen2016, Fodor2016, Murugan2017, Nardini2017, Nguyen2018}. Minimal models of active and driven systems provide analytical and numerically tractable test beds to investigate the interplay between dissipation and material properties far from equilibrium~\cite{Marchetti2013, Han2016, Bechinger2016, delJunco2018, Marchetti2018}. They have illustrated, for instance, how nonequilibrium driving can induce phase transitions and excite novel collective responses in soft media~\cite{Vicsek1995, Tailleur2008, Han2016, Nguyen2016, VanZuiden2016}. Recent theoretical work has proposed extending equilibrium concepts to active media, such as the definition of pressure~\cite{Takatori2015, Solon2015a, Solon2015b}, to rationalize their phenomenology. Others have strived to obtain stationary properties through perturbation close to equilibrium~\cite{Nardini2017}, inspired by previous approaches on driven systems~\cite{McLennan1959, Komatsu2008, Maes2009, Maes2010}.


To investigate how dissipation controls emerging behavior, yet another approach consists in biasing dynamical ensemble. Using large deviation techniques, trajectories are conditioned to promote atypical realizations of the dynamics. This has served, for instance, to investigate the role of dynamical heterogeneities in glassy systems~\cite{garrahan2007, Hedges2009, Pitard2011, Speck2012, Bodineau2012a, Limmer2014, Nemoto2017}, and soliton solutions in high-dimensional chaotic chains~\cite{tailleur2007probing, laffargue2013}. More recently, it has been shown that changing dissipation, through a dynamical bias, strongly affects the internal transport and the density fluctuations of nonequilibrium liquids~\cite{Cagnetta2017, nemoto2018optimizing}, confirming that dissipation control is indeed a fruitful route to tailoring material properties. Yet, despite these advances, any generic principle between dissipation and spatial organization still remains elusive.


In this paper, we first explore how dissipation affects the dynamics and structure of driven liquids. We consider in Sec.~\ref{sec:method} an assembly of $N$ Brownian particles where a subset $\Omega$ is driven by an external force ${\bf F}_{\rm d}$:
\begin{equation}\label{eq:dyn}
	\gamma\dot{\bf r}_i = \delta_{i\in\Omega}{\bf F}_{\rm d} - \nabla_i \sum_j v({\bf r}_i-{\bf r}_j) + {\boldsymbol\xi}_i ,
\end{equation}
where $\delta_{i\in\Omega}=1$ if $i\in\Omega$ and $\delta_{i\in\Omega}=0$ otherwise. The driven particles belonging to the set $\Omega$ are referred to as \textit{tracers}. We mainly focus on instances where the fraction of driven particles is less than the fraction of undriven particles, so that undriven particles are referred to as \textit{bath} particles. The fluctuating term ${\boldsymbol\xi}_i$ is a zero-mean white Gaussian noise with correlations $\langle\xi_{i\alpha}(t)\xi_{j\beta}(0)\rangle=2\gamma T\delta_{ij}\delta_{\alpha\beta}\delta(t)$, where $\gamma$ and $T$ respectively denote the damping coefficient and the bath temperature, with the Boltzmann constant set to unity $k_{\rm B}=1$. Building on recent work~\cite{Han2016, delJunco2018}, the driving force ${\bf F}_\text{d}$ is derived from a time-periodic protocol in two dimensions:
\begin{equation}\label{eq:theta}
	{\bf F}_{\rm d}(t) = f \big[\sin(\omega t) \hat{\bf e}_x + \cos(\omega t)\hat{\bf e}_y \big] ,
\end{equation}
where $f$ and $\omega$ are respectively the amplitude and the frequency of the drive, such as the drive persistence reads $\tau=2\pi/\omega$. The relative strength of the drive is given by the P\'eclet number $\text{Pe} = \sigma f/T$, where $\sigma$ is the typical particle size~\cite{Han2016, delJunco2018}. In the absence of interactions ($v=0$), the average position of driven tracers follows a periodic orbit, describing a circle in two dimensions.


Considering the diffusion coefficient of a tagged driven \textit{tracer} particle and the density correlations between tracer particles and undriven \textit{bath} particles in the liquid, we connect dissipation to liquid properties. At variance with~\cite{delJunco2018}, our analytic predictions are based on a systematic coarse-graining, thus providing an explicit dependence in terms of microscopic details~\cite{Dean1996, Demery2011, Demery2014}. Besides, based on numerical simulations, we also put forward an empirical relation between the deviation of tracer-bath correlations from equilibrium and the dissipation. Overall, these results clarify how nonequilibrium forces affect the transport and structure of the liquid, illustrating how liquid properties can be modified at the cost of energy dissipation.


To provide some concrete intuition on how to stabilize particular configurations with a nonequilibrium drive, we then investigate the emerging structure of Brownian particles subject to dynamical bias. In such {\it biased ensembles}, rare noise fluctuations are sampled to effectively drive the dynamics away from typical behavior. In practice, biased noise realizations can then serve as proxies for complex protocols, where energy is injected into particular modes by external forces, to control the dynamics. As a first step, we put forward in Sec.~\ref{sec:bias} a biased ensemble where trajectories indeed mimic the effect of an external drive. Using large deviation techniques~\cite{garrahan2007, Hedges2009, Jack2010, Pitard2011, Speck2012, Bodineau2012a, Chetrite2013, Limmer2014, Nemoto2017}, we obtain perturbatively an auxiliary dynamics where, at first order, trajectory bias is effectively realized by introducing a constant force. This supports the relevance of our approach to capture the behavior of driven systems with dynamical bias.


Then, we identify a class of biasing observables which all affect interactions in a controlled manner. As a proof of principle, we analyze biased trajectories for a specific choice in this class where, at any order, biasing can be simply rationalized as enhanced clustering with renormalized interactions. Direct sampling of such a biased ensemble, based on cloning algorithm~\cite{Giadina2006, tailleur2007probing, Hurtado2009, Nemoto2016, Ray2018, Klymko2018, Brewer2018}, confirms that tuning desired interactions changes the liquid structure through precise control, leading to the formation of specific \textcolor{blue}{structures $\to$ patterns}. This can potentially result in design principles, where pair-specific interactions are tailored, for the construction of complex addressable self-assembly landscapes.


% ===============================================================================


\section{Dissipation and liquid properties}\label{sec:method}

In this Section, we provide a series of generic relations between the energy dissipation, induced by periodically driven tracers, and the liquid properties. We consider the specific drive~\eqref{eq:theta} as a minimal case study, and we highlight below which results are expected to also hold in more general settings.


As a first step, the dynamics of undriven particles can be described in terms of a coarse-grained variable. Using standard techniques, the dynamics of the density field $\rho({\bf r},t) = \sum_{i\not\in\Omega}\delta[{\bf r}-{\bf r}_i(t)]$ can be written as a non-linear Langevin equation~\cite{Dean1996}. In the regime of weak interactions, the density fluctuations $\delta \rho({\bf r},t ) = \rho({\bf r}, t) - \rho_0$ around the average density $\rho_0$ are Gaussian in terms of the following Hamiltonian~\cite{Chandler1993, Demery2014, Kruger2017}:
\begin{equation}
	\begin{aligned}
		{\cal H} &= \frac{T}{2} \int \delta \rho ({\bf r}) K({\bf r} - {\bf r}') \delta \rho ({\bf r}') \dd{\bf r}\dd{\bf r}'
		\\
		&\quad + \int \sum_{i\in\Omega} v({\bf r}-{\bf r}_i) \rho({\bf r}) \dd{\bf r} ,
	\end{aligned}
\end{equation}
where $K({\bf r})= \delta({\bf r})/\rho_0 + v({\bf r})/T$. The conserved density dynamics reads
\begin{equation}\label{eq:EvolutionField}
	\begin{aligned}
		\frac{\partial \delta \rho({\bf r}, t)}{\partial t} &= D_{\rm G} \nabla^2 \int K({\bf r}-{\bf r}') \delta \rho({\bf r}', t) \dd{\bf r}'
		\\
		&\quad + \frac{1}{\gamma_{\rm G}} \nabla^2 \sum_{i\in\Omega}v({\bf r}-{\bf r}_i(t)) + \nabla \cdot {\boldsymbol\Lambda}({\bf r},t) ,
	\end{aligned}
\end{equation}
where $D_{\rm G} = \rho_0 T/\gamma$ and $\gamma_{\rm G}=\gamma/\rho_0$ are respectively the field diffusion coefficient and the field damping coefficient. The term $\boldsymbol\Lambda$ is a zero-mean Gaussian white noise with correlations $\langle \Lambda_\alpha({\bf r}, t) \Lambda_\beta({\bf r}', t') \rangle = 2 D_{\rm G} \delta_{\alpha\beta}\delta({\bf r} - {\bf r}')\delta(t-t')$.


Owing to the linearity of the density dynamics~\eqref{eq:EvolutionField}, it can be readily written in Fourier space $\delta\rho({\bf q},t)=\int\rho({\bf r},t){\rm e}^{-{\rm i}{\bf q}\cdot{\bf r}} {\rm d}{\bf r}$ as
\begin{equation}\label{eq:FourierEvolutionField}
	\begin{aligned}
		\frac{\partial \delta \rho({\bf q},t)}{\partial t} &= - |{\bf q}|^2 D_{\rm G} K({\bf q}) \delta \rho({\bf q},t)
		\\
		&\quad - |{\bf q}|^2 \frac{v({\bf q})}{\gamma_{\rm G}} \sum_{j\in\Omega}\ee^{-\ii {\bf q} \cdot {\bf r}_j(t)} + \ii{\bf q}\cdot{\boldsymbol\Lambda} ({\bf q},t) ,
	\end{aligned}
\end{equation}
so that the field dynamics can be directly solved as
\begin{equation}\label{eq:rho}
	\begin{aligned}
		\delta\rho({\bf q},t) &= \int_{-\infty}^t \dd s \ee^{-D_{\rm G} |{\bf q}|^2 K({\bf q})(t-s)}
		\\
		&\quad\times \bigg[ \ii{\bf q} \cdot {\boldsymbol\Lambda}({\bf q},s) - |{\bf q}|^2 \frac{v({\bf q})}{\gamma_{\rm G}} \sum_{j\in\Omega}\ee^{-\ii {\bf q}\cdot {\bf r}_j(s)}\bigg] .
	\end{aligned}
\end{equation}
Considering the limit of dilute driven tracers, where interactions among them are negligible, their dynamics reads
\begin{equation}\label{eq:EvolutionTracer}
	\gamma\dot{\bf r}_j = {\bf F}_\text{d} + {\boldsymbol\xi_j} - \int_{\bf q} \ii{\bf q} v(-{\bf q})\ee^{\ii{\bf q} \cdot {\bf r}_j(t)} \delta \rho({\bf q}, t) ,
\end{equation}
with $\int_{\bf q}=\int\dd{\bf q}/(2\pi)^d$ and $d$ referring to spatial dimension. As a result,~\eqref{eq:rho} and~\eqref{eq:EvolutionTracer} provide a  closed dynamics for tracers only. It should only be valid for weak interactions {\it a priori}, yet previous works have shown that it remains qualitatively relevant even beyond this regime in practice~\cite{Demery2015, Martin2018}. Indeed, Gaussian field theories for density fluctuations provide a very good description of simple liquids~\cite{Chandler1993}.


To characterize the transport properties of the liquid in the presence of the driving forces, our first goal is to obtain an explicit expression, in terms of microscopic details, for the tracer diffusion coefficient:
\begin{equation}
	D = \underset{t\to\infty}{\lim} \frac{1}{2dt} \big\langle \big[\langle{\bf r}_i(t)\rangle-{\bf r}_i(t)\big]^2 \big\rangle .
\end{equation}
We aim to explore connections between $D$ and dissipation, which is defined from stochastic thermodynamics as the power of the forces exerted by all tracers on solvent: ${\cal J} = \sum_{i\in\Omega}\langle\dot{\bf r}_i\cdot(\gamma\dot{\bf r}_i-{\boldsymbol\xi}_i)\rangle$, where here $\cdot$ denotes a Stratonovich product~\cite{Sekimoto1998, Seifert2012}. Substituting the dynamics~\eqref{eq:dyn}, the dissipation can be separated into free-motion and interaction contributions as ${\cal J} = f^2/\gamma - \dot w$, where the {\it rate of work} reads
\begin{equation}\label{eq:work}
	\dot w = \frac{1}{\gamma} \underset{j\not\in\Omega}{\sum_{i\in\Omega}}\big\langle {\bf F}_{\rm d}\cdot\nabla_iv({\bf r}_i-{\bf r}_j)\big\rangle.
\end{equation}
Note that we have used $\sum_{\{i,j\}\in\Omega}\nabla_iv({\bf r}_i-{\bf r}_j)=0$ to simplify $\dot w$ for a symmetric $v$. Given that $\dot w$ is the only non-trivial contribution to dissipation, connecting diffusion and dissipation simply amounts to expressing $D$ in terms of $\dot w$.


To set up a proper perturbation scheme, we scale the pair potential $v$ with a dimensionless parameter $h\ll1$, which controls the coupling between tracer and bath equations of motion. In Appendix~\ref{app:diff}, we obtain some explicit expressions for $D$ and $\dot w$ to quadratic order in $h$ together with various limits of external driving parameters, such as amplitude or frequency. 
We first consider the limit of high frequencies $\omega\tau_{\rm r}\gg 1$, where the relaxation time scale $\tau_{\rm r}=(D_{\rm G}/\sigma^2)K(|{\bf q}|=1/\sigma)$ is set here by density diffusion over tracer size $\sigma$, and of small P\'eclet number ${\rm Pe}\ll 1$. In this regime, the rate of work $\dot w$ and the increase of the diffusion coefficient $D-D_{\rm eq}$, where $D_{\rm eq}$ is the equilibrium diffusion, scale like
\begin{equation}\label{eq:small}
	\begin{aligned}
		\dot w &= \Big(\frac{h{\rm Pe}}{\omega}\Big)^2 \cdot \frac{(T/\sigma)^2}{d\gamma^3} \int_{\bf q} |{\bf q}|^4 |v({\bf q})|^2 \frac{1 + \rho_0K({\bf q})}{K({\bf q})} ,
		\\
		D-D_{\rm eq} &= \Big(\frac{h{\rm Pe}}{\omega}\Big)^2 \cdot \frac{T /\sigma^2}{d\gamma^3} \int_{\bf q}\frac{|{\bf q}|^2 |v({\bf q})|^2 } {K({\bf q})\big[1 + \rho_0 K({\bf q})\big]} .
	\end{aligned}
\end{equation}
These results are valid to quadratic order in $\rm Pe$ and in $h$. In the opposite limit, namely at low frequencies $\omega\tau_{\rm r}\ll 1$ and small P\'eclet number ${\rm Pe}\ll1$, we get
\begin{equation}\label{eq:large}
	\begin{aligned}
		\dot w &=  \frac{(h{\rm Pe})^2}{d\gamma\sigma^2} \int_{\bf q} \frac{|v({\bf q})|^2}{K({\bf q})\big[1 + \rho_0K({\bf q})\big]} ,
		\\
		D - D_{\rm eq} &= \frac{5(h{\rm Pe})^2}{d\gamma T\sigma^2} \int_{\bf q} \frac{|v({\bf q})|^2}{|{\bf q}|^2K({\bf q})\big[1 + \rho_0K({\bf q})\big]^3} .
	\end{aligned}
\end{equation}
Both $\dot w$ and $D-D_{\rm eq}$ are now independent of the driving frequency $\omega$. Again, this is valid to order $h^2$ and ${\rm Pe}^2$. Note that the scaled work $\gamma\dot w/f^2$ coincides with the reduced equilibrium diffusion $\gamma D_{\rm eq}/T-1$ to this order~\cite{Demery2011, Demery2014}, as expected from linear response. Our perturbation theory hence shows that scalings of $\dot w$ and $D-D_{\rm eq}$ are identical, both in terms of the drive amplitude $f$ and of its frequency $\omega$.


When the size $a$ of the bath particles is significantly smaller than the tracer size $\sigma\gg a$, which amounts to setting different pair potential $v$ for bath-bath and for bath-tracer interactions, one can safely neglect the variation of $K({\bf q})$ in~(\ref{eq:small}-\ref{eq:large}), such as $K({\bf q}) \simeq K(|{\bf q}|=1/a)$. Then, in both regimes $\omega\tau_{\rm r}\gg 1$ and $\omega\tau_{\rm r}\ll 1$, the renormalization of the diffusion coefficient $D-D_{\rm eq}$ can be simply written in terms of the rate of work $\dot w$ for ${\rm Pe}\ll1$ as
\begin{equation}\label{eq:work_D}
	\frac{D-D_{\rm eq}}{\sigma^2} \sim \frac{\dot w}{T} .
\end{equation}
Thus, the excess rate at which tracers move over their own size compared to equilibrium, set by the lhs of~\eqref{eq:work_D}, is controlled by the rate at which work is applied on tracers, set by the rhs of~\eqref{eq:work_D}. The proportionality factor is determined by the details of interactions and of the density fluctuations. This result corroborates some numerical observations obtained previously for a similar system, where composition-dependent diffusion constants can lead to phase transitions~\cite{delJunco2018}.


The specific case of a constant drive is already contained in the regime $\omega\tau_{\rm r}\ll1$. Besides, more general periodic protocols can also be addressed by considering a linear combination of the drive~\eqref{eq:theta} with arbitrary frequencies: ${\bf F}_{\rm d, gen}(t) = f \sum_n c_n\big[\sin(\omega_n t) \hat{\bf e}_x + \cos(\omega_n t)\hat{\bf e}_y \big]$. The results~(\ref{eq:small}-\ref{eq:work_D}) extend straightforwardly to these cases, as shown in Appendix~\ref{app:diff}, thus supporting the relevance of our approach for a generic periodic drive.


We now explore how dissipation can be connected to density correlations. To this end, we treat undriven bath particles without any approximation in what follows, instead of relying on the Gaussian density field theory for $\delta\rho$ as we did above. Using It\^o calculus, the average rate of change of the bath-tracer interaction potential $U = \sum_{i\in\Omega,j\not\in\Omega} v({\bf r}_i-{\bf r}_j)$ can be written as
\begin{equation}
	\gamma\langle\dot U \rangle = \underset{j\not\in\Omega}{\sum_{i\in\Omega}} \big\langle\big[\gamma(\dot{\bf r}_i - \dot{\bf r}_j) + 2 T \nabla_i\big] \circ \nabla_i v({\bf r}_i-{\bf r}_j)\big\rangle ,
\end{equation}
where here $\circ$ denotes an It\^o product. Substituting the dynamics~\eqref{eq:dyn} and using $\langle{\boldsymbol\xi}_i\circ\nabla_i v\rangle=0$, we get
\begin{equation}\label{eq:dotU}
	\begin{aligned}
		\gamma\langle&\dot U \rangle = \underset{j\not\in\Omega}{\sum_{i\in\Omega}} \big\langle({\bf F}_{\rm d} + 2 T \nabla_i) \cdot \nabla_i v({\bf r}_i-{\bf r}_j)\big\rangle
		\\
		& + \underset{j\not\in\Omega}{\sum_{i\in\Omega,k}} \big\langle \big[\nabla_i v({\bf r}_i-{\bf r}_j)\big]\cdot \nabla_k \big[ v({\bf r}_i-{\bf r}_k) - v({\bf r}_j-{\bf r}_k) \big] \big\rangle ,
    \end{aligned}
\end{equation}
where the $k$-sum runs over all particles. From the steady-state condition $\langle\dot U \rangle=0$, we then deduce that the rate of work $\dot w$, defined in~\eqref{eq:work}, can be written in terms of density correlations as
\begin{equation}\label{eq:balance}
	\begin{aligned}
		&\gamma\dot w = 2\rho_0 \int g({\bf r}) \big[(\nabla v({\bf r}))^2 - T \nabla^2 v({\bf r})\big] \dd{\bf r} ,
		\\
		&\, + \rho_0^2 \iint\big[ g_{3a}({\bf r}, {\bf r}') + g_{3b}({\bf r}, {\bf r}')\big]  \big[\nabla v({\bf r})\big] \cdot \big[\nabla v({\bf r}')\big] \dd{\bf r} \dd{\bf r}' ,
	\end{aligned}
\end{equation}
where
\begin{equation}
	\begin{aligned}
 		g({\bf r}) &= \frac{1}{N} \underset{j\not\in\Omega}{\sum_{i\in\Omega}} \big\langle \delta({\bf r}-{\bf r}_i+{\bf r}_j)\big\rangle ,
		\\
 		g_{3a}({\bf r},{\bf r}') &= \frac{1}{N^2} \underset{j\not\in\Omega}{\sum_{i\in\Omega,k}}' \big\langle \delta({\bf r}-{\bf r}_i+{\bf r}_j) \delta({\bf r}'-{\bf r}_i+{\bf r}_k)\big\rangle ,
		\\
 		g_{3b}({\bf r},{\bf r}') &= \frac{1}{N^2} \underset{j\not\in\Omega}{\sum_{i\in\Omega,k}}' \big\langle \delta({\bf r}-{\bf r}_i+{\bf r}_j) \delta({\bf r}'-{\bf r}_j+{\bf r}_k)\big\rangle ,
	\end{aligned}
\end{equation}
and $\sum'$ denotes a sum without the overlap of indices: $k\neq i$ and $k\neq j$. The power balance~\eqref{eq:balance} reflects how density correlations adapt to the presence of nonequilibrium forces. For a vanishing rate of work ($\dot w = 0$), one recovers the first order of the equilibrium Yvon-Born-Green hierarchy, in its integral form, for two-component fluids~\cite{Hansen2013}. At finite rate of work ($\dot w\neq0$), the relation between the two-body correlation $g$ and the three-body terms $\{g_{3a},g_{3b}\}$ is now implicitly constrained by dissipation. This is our first main result, which is valid for an arbitrary external drive.


Though being an exact result, it is not straightforward to test, either numerically or experimentally, due to the three-body correlations. In equilibrium, where tracer and bath particles are indistinguishable, we get $g_{3a}=g_{3b}$. Assuming that this remains approximately valid in the driven case for a dilute fraction of tracers, the rate of work can simply be written in terms of the force exerted by bath particles on a tracer ${\bf F}_i =- \sum_{j\not\in\Omega}\nabla_iv({\bf r}_i-{\bf r}_j)$ as
\begin{equation}\label{eq:approx}
	\gamma\dot w \simeq 2\sum_{i\in\Omega} \big[ \big\langle{\bf F}_i^2\big\rangle + T \big\langle\nabla_i\cdot{\bf F}_i\big\rangle \big] ,
\end{equation}
where we have used $\sum_i\simeq\sum_{i\not\in\Omega}$ in the dilute limit. We simulate the dynamics~\eqref{eq:dyn} with the WCA potential $v({\bf r})=4v_0\big[(\sigma/|{\bf r}|)^{12}-(\sigma/|{\bf r}|)^6\big]\Theta(2^{1/6}\sigma-|{\bf r}|)$, where $\Theta$ denotes the Heaviside step function~\cite{WCA1971}. Our measurements in Fig.~\ref{fig:fig1} show that~\eqref{eq:approx} is indeed a good approximation at small $\rm Pe$ and small $\tau$, namely when the drive only weakly perturbs the liquid. Besides, up to ${\rm Pe}=36$ and $\tau/\tau_{\rm r}=50$, the relative deviation is at most of about $10\%$. At variance with previous approaches~\cite{Harada2005, Lander2012, Battle604}, which rely on probing the whole system, our results demonstrate that dissipation can actually be evaluated, with only a small error, by considering solely bath-tracer forces: the contribution of interaction forces among other particles is negligible for a dilute fraction of driven tracers.


\begin{figure}
	\centering
	\includegraphics[width=\linewidth]{fig1.pdf}
	\caption{\label{fig:fig1}
		Parametric plot of the rate of work $\dot w/T$ and the statistics of bath-tracer forces $\sum_{i\in\Omega}\big[\langle{\bf F}_i^2\rangle + T\langle\nabla_i\cdot{\bf F}_i\rangle\big]/\gamma T$, where the sum runs over driven tracer particles, for a dilute fraction of driven particles ($10\%$). $\tau/\tau_{\rm r}=20$ (black), $30$ (brown), $40$ (red), $50$ (orange). ${\rm Pe}=6$ ($\hexagon$), $12$ ($\boxempty$), $18$ ({\large$\vartriangle$}), $24$ ({$\Circle$}), $30$ (${\Diamond}$), $36$ ($\varhexagon$). The good agreement with the approximate relation~\eqref{eq:approx}, in solid line, indicates that dissipation can be estimated by only measuring bath-tracer forces.
		Simulation details in Appendix~\ref{app:simu}.
	}
\end{figure}


To further evaluate the change in liquid structure induced by dissipation, we measure the deviation from equilibrium pair-correlation $g-g_{\rm eq}$ due to the driving forces. At a given $\tau/\tau_{\rm r}$, scaling ${\cal I} = \big[(\nabla v)^2-T\nabla^2v\big](g - g_{\rm eq})$ by ${\rm Pe}^2$ reveals that all curves almost collapse into a master curve for our numerical range ${\rm Pe}\in[12,36]$, as shown in the middle column of Fig.~\ref{fig:fig2}. Given that the rate of work also exhibits such a scaling, this suggests the existence of an underlying relation between $\int{\cal I}({\bf r}){\rm d}{\bf r}$ and $\dot w$: it appears that their ratio indeed gives a factor close to $\rho_0/\gamma$, as depicted in the right column of Fig.~\ref{fig:fig2}. Hence, this empirical relation shows that, considering the periodic drive~\eqref{eq:theta} in the dilute limit, the dissipation can be directly estimated by comparing driven and equilibrium pair-correlations: this is clearly an asset with respect to invasive methods based on comparing fluctuations and response~\cite{Harada2005, Mizuno2007, Visco2015, Turlier2016, Ahmed2018}.


\begin{figure*}
	\centering
	\includegraphics[width=\linewidth]{fig2a.pdf}
	\vskip.5cm
	\includegraphics[width=\linewidth]{fig2b.pdf}
	\vskip.5cm
	\includegraphics[width=\linewidth]{fig2c.pdf}
	\vskip.5cm
	\includegraphics[width=\linewidth]{fig2d.pdf}
	\caption{\label{fig:fig2}
		(Left) Bath-tracer density correlation $g$ as a function of inter-particle distance $r/\sigma$. Blue solid line is equilibrium correlation $g_{\rm eq}$ for ${\rm Pe}=0$.
		(Middle) ${\cal I} = \big[(\nabla v)^2-T\nabla^2v\big](g - g_{\rm eq})$ scaled by ${\rm Pe}^2$ as a function of $r/\sigma$. For a given $\tau/\tau_{\rm r}$, the data almost collapse into a master curve.
		(Right) $(\rho_0/\gamma T)\int{\cal I}({\bf r}){\rm d}{\bf r}$ and rate of work $\dot w/T$ as functions of $\rm Pe$. The former under-estimates $\dot w$ by at most approximately $10\%$, showing that deviation from equilibrium pair correlations provides a direct access to dissipation with only a small error.
		Simulation details in Appendix~\ref{app:simu}.
	}
\end{figure*}


Overall, the results of this Section illustrate how dissipation affects the transport properties of a driven liquid, measured in terms of diffusion coefficient, as well as its structural properties, characterized by density fluctuations. This immediately suggests that internal transport and emerging structure can be tuned externally by monitoring the nonequilibrium forces which power the dissipation. A related question then is whether specific structures and correlations can be stabilized by biasing particle trajectories to mimic the effect of generic external drive. We now investigate this question. 


% ===============================================================================


\section{Interactions in biased ensembles}\label{sec:bias}

In this Section, we explore how to promote atypical structures in a system of interacting Brownian particles with dynamics
\begin{equation}\label{eq:dyn_eq}
	\gamma\dot{\bf r}_i = - \nabla_i \sum_j v({\bf r}_i-{\bf r}_j) + {\boldsymbol\xi}_i ,
\end{equation}
where the statistics of the noise term ${\boldsymbol\xi}_i$ is the same as to the one in~\eqref{eq:dyn}. At variance with Sec.~\ref{sec:method}, we aim to change the emerging collective behavior by means of a dynamical bias, instead of an external force, where trajectories are now driven by some atypical realizations of the noise. To sample such a {\it biased ensemble}, we rely on the framework of large deviation theory~\cite{Chetrite2013, Jack2010}. Formally, it is generated by introducing an exponential weighting factor $\exp\big[\kappa\int_0^t{\cal E}(s){\rm d}s\big]$ in the dynamical path probability, where $\cal E$ and $\kappa$ respectively denote a given observable and its associated control parameter. The trajectories and configurations in this biased probability are then regarded as the large-deviation realizations conditioned by atypical values of $\cal E$, which can be tuned by changing $\kappa$. Our first goal is to show that trajectories in some appropriate biased ensembles  can be effectively equivalent to the ones generated by an ensemble driven by an external force.


Inspired by the crucial role of dissipation in emerging liquid properties, as discussed in Sec.~\ref{sec:method}, we consider biasing the equilibrium dynamics with the stochastic  rate of \textcolor{blue}{\it virtual} work that would be produced by applying a constant force ${\bf F}_{\rm d}$ to a subset $\Omega$ of particles:
\begin{equation}\label{eq:eps}
	{\cal E} = \frac{1}{\gamma T}\sum_{i\in\Omega} {\bf F}_{\rm d}\cdot\nabla_i V ,
\end{equation}
where $V=(1/2)\sum_{i,j}v({\bf r}_i-{\bf r}_j)$. Following the procedure in~\cite{Jack2010,Chetrite2013}, an auxiliary physical dynamics, with the same statistical properties as in the biased ensemble, can be constructed by solving the following eigenvalue equation 
\begin{equation}\label{eq:adjointbiased}
	\big[L^\dagger + \kappa{\cal E}\big]\,{\cal G}(\{{\bf r}_i\},\kappa) = \lambda(\kappa)\,{\cal G}(\{{\bf r}_i\},\kappa) ,
\end{equation}
where the eigenvalue $\lambda$, parametrized by $\kappa$, is the scaled cumulant generating function appropriate to the biasing observable $\cal E$. The operator $L^\dagger$ is the adjoint of the Fokker-Planck operator associated with the equilibrium dynamics~\eqref{eq:dyn_eq}:
\begin{equation}\label{eq:L}
	L^\dagger = \frac{1}{\gamma} \sum_i\big[ T \nabla_i - \nabla_i V\big] \cdot\nabla_i .
\end{equation}
The potential $\tilde V$ of the auxiliary dynamics is then defined in terms of the eigenfunction ${\cal G}$ as 
\begin{equation}
	\tilde V = V - 2 T \ln{\cal G} .
\end{equation}
In practice, computing $\tilde V$ is a highly non-trivial procedure for many-body systems. The explicit solutions obtained so far for particle-based dynamics are mostly restricted to non-interacting systems~\cite{Chetrite2013, Touchette2016}.


In our case, a simple expression can be obtained for the auxiliary potential $\tilde V$ by solving~\eqref{eq:adjointbiased} perturbatively in terms of the bias parameter $\kappa$. Specifically, we expand
\begin{equation}
	\begin{aligned}
		\lambda(\kappa) &= \kappa \langle{\cal E}\rangle + {\cal O}(\kappa^2) ,
		\\
		{\cal G}(\{{\bf r}_i\},\kappa) &= {\cal G}_0 + \kappa\,{\cal G}_1(\{{\bf r}_i\}) + {\cal O}(\kappa^2) ,
	\end{aligned}
\end{equation}
where ${\cal G}_0$ is the uniform eigenvector associated with the zero eigenvalue. Given that $\langle{\cal E}\rangle=0$ in steady state, which follows from the vanishing current condition in the unbiased dynamics ($\langle\dot{\bf r}_i\rangle=0$), the leading non-trivial order of~\eqref{eq:adjointbiased} reads
\begin{equation}
	\kappa\big[L^\dagger{\cal G}_1 + {\cal G}_0 {\cal E}\big] + {\cal O}(\kappa^2) = 0 .
\end{equation}
Substituting the explicit expressions for the biasing function~\eqref{eq:eps} and for the operator~\eqref{eq:L}, we then deduce that $T{\cal G}_1 = {\cal G}_0 \sum_{i\in\Omega}{\bf F}_{\rm d}\cdot{\bf r}_i$ is a solution of the eigenvalue problem to order $\kappa$. The auxiliary potential follows as
\begin{equation}\label{eq:V}
	\tilde V = V - 2\kappa \sum_{i\in\Omega} {\bf F}_{\rm d}\cdot{\bf r}_i + {\cal O}(\kappa^2) .
\end{equation}
As a result, the trajectories biased by~\eqref{eq:eps} can be generated with a physical dynamics where, at leading order, one applies the external force $2\kappa{\bf F}_{\rm d}$ to every particle in $\Omega$. In particular, this potential does not include any long-range interactions which are usually found in auxiliary dynamics~\cite{Jack2015}. This establishes the equivalence between our dynamical bias and an external weak driving, at the level of trajectories, suggesting that biased ensembles indeed provide a relevant framework to explore deeper the effect of nonequilibrium forces.


We then consider the effect of higher-order bias and show that such a higher-order effect can be anticipated by considering another biasing observable,
\begin{equation}\label{eq:eps_prime}
	\varepsilon = \frac{\kappa^2}{\gamma T}\sum_{i\in\Omega}\int_0^t{\bf F}_{\rm d}^2{\rm d}s, 
\end{equation}
in the auxiliary physical dynamics given by the potential~\eqref{eq:V}. Biasing with $\varepsilon$ the auxiliary dynamics at first order, the corresponding path probability $\cal P$ can be written from standard methods~\cite{Martin1973, Dominicis1975}. Specifically, introducing the Lagrangian ${\cal L}_i$ so that ${\cal P}\sim\exp\big[-\sum_i\int_0^t{\cal L}_i(s){\rm d}s\big]$, it reads
\begin{equation}
	{\cal L}_i = \frac{1}{4\gamma T} \big[\gamma\dot{\bf r}_i - 2\kappa\delta_{i\in\Omega}{\bf F}_{\rm d} + \nabla_iV\big]^2 - \frac{1}{2\gamma} \nabla^2_iV - \frac{\kappa^2}{\gamma T}\delta_{i\in\Omega}{\bf F}_{\rm d}^2 ,
\end{equation}
and, expanding the first term, we get
\begin{equation}
	{\cal L}_i = \frac{1}{4\gamma T} \big[\gamma\dot{\bf r}_i + \nabla_iV\big]^2 - \frac{1}{2\gamma} \nabla^2_iV - \frac{\kappa}{\gamma T} \delta_{i\in\Omega}{\bf F}_{\rm d}\cdot\nabla_i V ,
\end{equation}
up to a boundary term which is negligible at large times ($t\to\infty$). Hence, it appears that $\cal P$ coincides with the path probability of the original dynamics biased with~\eqref{eq:eps}. Therefore, the effect of such a bias is actually identical, at any order, to introducing an external force and biasing with a constant term symmetric in $\kappa$, as given respectively in~\eqref{eq:V} and~\eqref{eq:eps_prime}.


Such a decomposition, in terms of first-order auxiliary potential and higher-order symmetric bias, can be extended to a generic class of observables. We now examine the case where the control parameters $\kappa_{ij}$ are specific for any particle pairs $\{i,j\}$, so that the biasing weight in path probability reads $\exp\big[\sum_{i,j}\kappa_{ij}\int_0^t{\cal E}_{ij}(s){\rm d}s\big]$, where ${\cal E}_{ij}$ is defined as
\begin{equation}\label{eq:eps_bis}
	{\cal E}_{ij} = \frac{1}{T} L^\dagger\big[ A({\bf r}_i-{\bf r}_j)\big] ,
\end{equation}
for an arbitrary observable $A$. With this definition, the property $T\langle{\cal E}_{ij}\rangle = \langle\dot A({\bf r}_i-{\bf r}_j)\rangle = 0$ allows one to follow the same procedure as previously. As detailed in Appendix~\ref{app:far}, biasing the equilibrium dynamics~\eqref{eq:dyn_eq} with~\eqref{eq:eps_bis} is now equivalent to a dynamical bias on the first-order auxiliary dynamics, given by the potential $V+2\sum_{i,j}\kappa_{ij}A({\bf r}_i-{\bf r}_j)$, with respect to
\begin{equation}\label{eq:eps_pprime}
	\varepsilon' = \frac{1}{\gamma T}\int_0^t\sum_k\Big[\sum_{i,j}\kappa_{ij} \nabla_kA({\bf r}_i(s)-{\bf r}_j(s))\Big]^2{\rm d}s .
\end{equation}
This is our second main result. In practice, it implies that interactions can be tailored at will for a weak bias, since it simply amounts to introducing an additional contribution to the bare potential. Besides, higher-order effects are all captured by maximizing the corresponding squared forces, as they appear in the integrand of~\eqref{eq:eps_pprime}.


To illustrate the potential of our bias to control liquid properties, we focus in what follows on the specific case where $A$ equals the pair interaction potential of the unbiased system $v$. With this choice, the bias $\gamma T{\cal E}_{ij} = \sum_k\big[T\nabla_k - \nabla_kV\big]v({\bf r}_i-{\bf r}_j)$ is reminiscent of, yet qualitatively different from, the escape rate used to promote dynamical heterogeneity in glassy systems~\cite{Pitard2011, Fullerton2013}. In our case, it results in two combined effects: (i)~renormalizing pair-specific interactions as $v({\bf r}_i-{\bf r}_j)\to[1+\kappa_{ij}]\,v({\bf r}_i-{\bf r}_j)$, and (ii)~optimizing the total squared force in the integrand of $\varepsilon'$, which effectively tends to cluster particles. Importantly, clustering is favored for both signs of the bias. This is in contrast with the emergence of a hyperuniform phase, where large scale fluctuations are suppressed, generally associated with biasing some hydrodynamic theories of diffusive systems~\cite{Jack2015b}.


Moreover, we choose $\kappa_{ij}=\kappa\delta_{i\in\Omega}\delta_{j\not\in\Omega}$, so that all the pairs between a subset $\Omega$ and other particles are biased with the same strength $\kappa$. To connect with the settings in Sec.~\ref{sec:method}, the set $\Omega$ could for instance refer to the tracer particles. In practice, modifying the interaction strength between targeted pairs is qualitatively consistent with the effect of external driving. For instance, phase separation in mixtures of driven and undriven particles, reported both experimentally and numerically, can be rationalized in terms of an effective decrease of specific interactions between these particles~\cite{delJunco2018,Han2016}.


\begin{figure*}
	\centering
	\includegraphics[width=.49\linewidth]{fig3a.pdf}
	\hfill
	\includegraphics[width=.49\linewidth]{fig3b.pdf}
	\vskip.5cm
	\includegraphics[width=.49\linewidth]{fig3c.pdf}
	\hfill
	\includegraphics[width=.49\linewidth]{fig3d.pdf}
	\caption{\label{fig:energybias}
		(a-b)~Average biased observable $\sum_{i\in\Omega,j\not\in\Omega}\langle{\cal E}_{ij}\rangle_\kappa$, defined in~\eqref{eq:eps_bis} where here $A=v$, as a function of bias parameter $\kappa$: first-order auxiliary dynamics and direct sampling of the biased ensemble respectively in circles and solid lines.
		(c-d)~Biased density correlation $g_\kappa$ as a function of inter-particle distance $r/\sigma$: auxiliary dynamics and direct sampling respectively in solid and dotted lines.
		Biasing modifies the potential $v$ for pairs $\{i\in\Omega,j\not\in\Omega\}$ by a factor $\kappa$ at leading order. The good agreement between auxiliary dynamics and direct sampling illustrates the control of liquid structure at small $\kappa$.
		Simulation details in Appendix~\ref{app:simu}.
}
\end{figure*}


To confirm numerically the validity of our approach, we first probe the range of the perturbative auxiliary dynamics, where interactions are simply renormalized. To this aim, we compare measurements of $\sum_{i\in\Omega, j\in\Omega}\langle{\cal E}_{ij}\rangle_\kappa$, where here $\langle\cdot\rangle_\kappa$ denotes an average in the biased ensemble, obtained from simulations of the first-order auxiliary dynamics and from a direct sampling of the biased ensemble. The latter is implemented with a cloning algorithm where rare realizations are regularly selected and multiplied for efficient sampling~\cite{Giadina2006, tailleur2007probing, Hurtado2009, Nemoto2016, Ray2018, Klymko2018, Brewer2018}. For convenience, interactions are now given by the soft-core potential $v({\bf r}) = v_0\exp\big[-1 / (1 - (|{\bf r}|/\sigma)^2)\big]\Theta(\sigma-|{\bf r}|)$. We observe a very good agreement between the two measurements for a finite range of $\kappa$. In particular, the agreement improves as $v_0/T$ is decreased, as reported in Figs.~\ref{fig:energybias}(a-b). This supports the validity of our perturbation in this regime, up to interaction change of about $+40\%$ when $v_0/T=4$. Note that the range of validity is asymmetric in $\kappa$. 


To explore further the features of the biased ensemble, we now compare the density correlations of biased pairs $g_\kappa({\bf r})\sim \sum_{i\in\Omega,j\not\in\Omega}\langle \delta({\bf r}-{\bf r}_i+{\bf r}_j)\rangle_\kappa$ obtained from both direct sampling and auxiliary dynamics. At $\kappa=\pm0.1$, the agreement is good for the whole curve when $v_0/T=4$, whereas a clear deviation appears beyond $r\simeq\sigma$ when $v_0/T=12$, as shown in Figs.~\ref{fig:energybias}(c-d). In both cases, the region of particle overlap $r<\sigma$ is well reproduced, as expected, since it is entirely controlled by the interaction change captured by the first-order auxiliary dynamics. Yet, the tendency for particles to cluster, manifest in the numerical peak at $r\simeq\sigma$, comes as a higher-order effect beyond this perturbation, and therefore is missed by $g_\varepsilon$ of the auxiliary dynamics. Note that the peak value is comparable for $\kappa=\pm0.1$, in agreement with~\eqref{eq:eps_pprime} being symmetric in $\kappa$. Besides, at fixed $\kappa$, the structural modification becomes more dramatic as $v_0/T$ increases. Altogether, these results demonstrate our bias indeed modulates the liquid structure in a controlled manner for small bias and weak interactions.


\begin{figure*}
	\centering
	\includegraphics[width=.9\linewidth]{fig4.pdf}
	\caption{\label{fig:outofperturbation}
	Configurations in biased ensemble obtained from direct sampling: biased pair potentials are the ones between blue and red particles. Blue and orange arrows respectively indicate (i) the increased propensity to form clusters for $\kappa=\pm3$, and (ii) the increase of pair interaction strength from $\kappa=-3$ to $3$. As a result, particles aggregate either with a random composition ($\kappa=3$) or in a micelle-like structure with a blue core ($\kappa=-3$), showing that pair interaction bias can favor some selected structures.
		Simulation details in Appendix~\ref{app:simu} and movies in~\cite{movie}.
	}
\end{figure*}


Finally, to determine whether our bias can stabilize configurations which are significantly distinct from the unbiased system, we probe numerically the effect of large bias $|\kappa|>1$ from direct sampling. The particles spontaneously tend to cluster for both positive and negative $\kappa$, as shown in Fig.~\ref{fig:outofperturbation} and movies in~\cite{movie}. This confirms the propensity of trajectories to maximize interaction forces at high bias, in agreement with~\eqref{eq:eps_pprime}. Importantly, the shape of clusters differs depending on the sign of $\kappa$: a micelle-like structure featuring the particles in $\Omega$ at the core (blue) surrounded by others (red) appears for $\kappa=-3$, whereas clusters have a random composition for $\kappa=3$. Again, this agrees with the renormalized interactions being either increased or decreased, respectively for $\kappa>0$ and $\kappa<0$. Overall, these results establish a reliable proof of principle for the design of self-assembled structures based on biasing energy flows into or out of particular pair interactions.


% ===============================================================================


\section{Conclusion}

Developing techniques to characterize and control the behavior of systems operating far from equilibrium remains a central and outstanding problem. In this paper, we have demonstrated that specifying the amount of energy dissipated by nonequilibrium forces allows one to constrain the dynamics and structure of such systems. Despite the apparently complex interplay between internal dissipation and emerging properties, their exist simple connections between tracer diffusion, density correlations and dissipation in a driven liquid. It remains to investigate whether similar results can be obtained in more complex systems, including for instance anisotropic building blocks, such as active liquid crystals or driven chiral objects~\cite{Joshi2017, VanZuiden2016, Nguyen2014b}. This would pave the way towards controlling accurately the emerging properties in such systems, by tuning correctly the amount of dissipated energy.


In practice, monitoring the dissipation through a unique parameter remains an open challenge for many-body systems, since it generally depends on the complex interplay between interactions and external driving. At variance, biased ensembles enable one to specify the statistics of dissipation by introducing an additional control parameter, analogously to a change from micro-canonical to canonical ensemble in equilibrium thermodynamics~\cite{Chetrite2013, Jack2010}. This is done by selecting rare noise realizations which drive the system away from its typical behaviour, without introducing any external force. Pioneer works were focused on favoring dynamical heterogeneities in kinetically constrained models~\cite{garrahan2007, Hedges2009, Pitard2011, Speck2012, Bodineau2012a}, without affecting the structure, more recent studies have shown the potential to also change density correlations~\cite{Jack2014, Cagnetta2017, nemoto2018optimizing}. Yet, anticipating the emergent biased dynamics and structure is still challenging in the presence of many-body interactions~\cite{Chetrite2013, Jack2010}, so that any precise control in this context has remained elusive so far.


Using large deviations techniques, we have put forward a particular set of biased ensembles which allows one to regulate the liquid structure in a controlled manner. At leading order, any bias in this class simply leads to introducing additional interactions in the dynamics. Importantly, we have demonstrated that higher order bias systematically constrain the trajectories to maximize squared forces induced by the first-order bias. Based on a minimal case study, we have sampled the biased configurations with state-of-the-art numerics~\cite{Giadina2006, tailleur2007probing, Hurtado2009, Nemoto2016, Ray2018, Klymko2018, Brewer2018}, thus illustrating the ability to stabilize specific structures in a controlled manner.


It is important to realize that dynamical biasing consists in favoring rare noise fluctuations to promote atypical configurations. Hence, the biased dynamics effectively provides useful insights on how to reproduce such configurations with an external drive: it should simply mimic the corresponding noise realizations. In practice, this line of thought has already been exploited for efficient sampling of the biased ensemble~\cite{Nemoto2016, Jack2017, Jack2018}. One can introduce external control forces as a way to make rare configurations become typical. The control is optimized, which can be formulated as a proper variational problem~\cite{Nemoto2011, Chetrite2015, Jack2015}, depending on the state to be promoted.


Finally, our analytic framework encompasses the case of a specific bias for each pair of particle. Then, it could be used as a fruitful route to promote the spontaneous self-assembly of complex structures. For instance, inspired by recent works~\cite{Murugan2015, Murugan2017b}, one might potentially design appropriate energetic landscapes, in terms of the pair-specific bias parameters, to selectively stabilize some target molecules. Moreover, it would be interesting to extend our results to active systems~\cite{Marchetti2013, Cates2015, Bechinger2016, Marchetti2018}, where the driving force has now an independent dynamics for each particle, as a way to open the door to controlling self-organization in biological context~\cite{Sperandio2002, Brown2011, Betz2018}.


% ===============================================================================


\acknowledgements{The authors acknowledge insightful discussions with Michael E. Cates, Robert L. Jack, and Vincenzo Vitelli. This work was granted access to the HPC resources of CINES/TGCC under the allocation 2018-A0042A10457 made by GENCI and of MesoPSL financed by the Region Ile de France and the project Equip@Meso (reference ANR-10-EQPX-29-01) of the program Investissements d'Avenir supervised by the Agence Nationale pour la Recherche. SV and LT were supported by the University of Chicago Materials Research Science and Engineering Center, which is funded by National Science Foundation under award number DMR-1420709. SV acknowledges support from the Sloan Foundation and startup funds from the University of Chicago. \'EF benefits from an Oppenheimer Research Fellowship from the University of Cambridge, and a Junior Research Fellowship from St Catherine's College.}


% ===============================================================================


\appendix

\section{Dissipation and diffusion}\label{app:diff}

This appendix is devoted to the derivation of the dissipation rate $\cal J$ and the diffusion coefficient $D$ of a driven tracer, as defined in Sec.~\ref{sec:method}. We employ a perturbative treatment at weak interactions, originally introduced for a particle driven at constant force in~\cite{Demery2011, Demery2014}. To this aim, the tracer-bath interaction potential $v$ is scaled by a small dimensionless parameter $h\ll1$ in what follows. Besides, we focus on the regime of dilute tracers, so that interactions among them, either direct or mediated by the bath, can be safely neglected .


The dynamic action associated with the tracer dynamics~(\ref{eq:rho}-\ref{eq:EvolutionTracer}) follows from standard path integral methods~\cite{Martin1973, Dominicis1975}. It can be separated into contributions from the free tracer motion and from interactions, respectively denoted by ${\cal A}_0$ and ${\cal A}_{\rm int}$:
\begin{equation}\label{eq:action}
	\begin{aligned}
		{\cal A}_0 &= \int \bar{\bf r}_0 \cdot \big[ {\rm i}(\dot{\bf r}_0 - {\bf F}_{\rm d}/\gamma) + D_0 \bar{\bf r}_0 \big] {\rm d}t ,
		\\
		{\cal A}_{\rm int} &= \frac{h^2}{\gamma} \int \frac{{\rm d}{\bf q}}{(2\pi)^d} |{\bf q}|^2 |v({\bf q})|^2 \int_{-\infty}^\infty {\rm d}s\int_{-\infty}^s{\rm d}u
		\\
		&\quad\times {\rm e}^{-D_{\rm G}|{\bf q}|^2K({\bf q})(s-u)+{\rm i}{\bf q}\cdot[{\bf r}_0(s)-{\bf r}_0(u)]}
		\\
		&\quad\times \bar{\bf r}_0(s) \cdot \bigg[ \frac{\bar{\bf r}_0(u)}{\gamma K({\bf q})} - \frac{{\bf q}}{\gamma_{\rm G}} \bigg] ,
	\end{aligned}
\end{equation}
where $D_0=T/\gamma$ is the tracer diffusion coefficient in the absence of interactions ($v=0$), and $\bar{\bf r}_0$ is the process conjugated with the tracer position ${\bf r}_0$. For weak interactions $h\ll1$, any average value can be then expanded in terms of $h$ as $\langle\cdot\rangle=\langle\cdot\rangle_0 - h^2 \langle{\cal A}_{\rm int}\cdot\rangle_0 + {\cal O}(h^4)$, where $\langle\cdot\rangle_0$ is the average taken with respect to ${\cal A}_0$ only. As a result, determining the first correction from interactions in any observable amounts to computing the corresponding average $\langle{\cal A}_{\rm int}\cdot\rangle_0$.


Considering the dissipation rate ${\cal J}=\langle\dot{\bf r}_0\rangle\cdot{\bf F}_{\rm d}$, the leading order is $\langle\dot{\bf r}_0\rangle_0 \cdot {\bf F}_{\rm d} = |{\bf F}_{\rm d}|^2/\gamma = f^2/\gamma$, and the first correction reads $ - h^2 \langle{\cal A}_{\rm int}\dot{\bf r}_0\rangle_0 \cdot {\bf F}_{\rm d} $. Given the explicit form of ${\cal A}_{\rm int}$ in~\eqref{eq:action}, the correlations of interest are
\begin{equation}
	\begin{aligned}
		\Big\langle\dot{\bf r}_0(t) &\big[{\bf q}\cdot\bar{\bf r}_0(s)\big] {\rm e}^{{\rm i}{\bf q}\cdot[{\bf r}_0(s)-{\bf r}_0(u)]}\Big\rangle_0
		\\
		&= {\rm i}{\bf q}\delta(t-s){\rm e}^{ - D_0|{\bf q}|^2(t-u) + \frac{{\rm i}{\bf q}}{\gamma}\cdot\int_u^t {\bf F}_{\rm d}(w) {\rm d}w} ,
		\\
		\Big\langle\dot{\bf r}_0(t) &\big[\bar{\bf r}_0(u)\cdot\bar{\bf r}_0(s)\big] {\rm e}^{{\rm i}{\bf q}\cdot[{\bf r}_0(s)-{\bf r}_0(u)]}\Big\rangle_0
		\\
		&= -{\rm i}{\bf q}\delta(t-s){\rm e}^{ - D_0|{\bf q}|^2(t-u) + \frac{{\rm i}{\bf q}}{\gamma}\cdot\int_u^t {\bf F}_{\rm d}(w) {\rm d}w} ,
	\end{aligned}
\end{equation}
where we have used that the tracer statistics is Gaussian in the absence of interactions, following~\cite{Demery2011, Demery2014}. From this result, we get
\begin{equation}
	\begin{aligned}
		&{\cal J} - f^2/\gamma
		\\
		&\,= \frac{h^2}{d\gamma^2} \int \frac{{\rm d}{\bf q}}{(2\pi)^d} {\rm i}{\bf q}\cdot{\bf F}_{\rm d}(t) |{\bf q}|^2 |v({\bf q})|^2 \frac{D_0+D_{\rm G}K({\bf q})}{D_0K({\bf q})}
		\\
		&\,\quad\times \int_{-\infty}^t {\rm d}u {\rm e}^{-|{\bf q}|^2 [D_0+D_{\rm G} K({\bf q})](t-u) + \frac{{\rm i}{\bf q}}{\gamma}\cdot\int_u^t{\bf F}_{\rm d}(w){\rm d}w}
		\\
		&\,\quad + {\cal O}(h^4) ,
	\end{aligned}
\end{equation}
where we have used $\gamma_{\rm G}=\gamma/\rho_0$ and $D_{\rm G}=\rho_0D_0$. Expanding at small $f$, we deduce
\begin{equation}
	\begin{aligned}
		&{\cal J} - f^2/\gamma
		\\
		&\,= - \frac{h^2}{d\gamma^3} \int \frac{{\rm d}{\bf q}}{(2\pi)^d} |{\bf q}|^4 |v({\bf q})|^2 \frac{D_0+D_{\rm G}K({\bf q})}{D_0K({\bf q})}
		\\
		&\,\quad\times \int_{-\infty}^t {\rm d}u {\rm e}^{-|{\bf q}|^2 [D_0+D_{\rm G} K({\bf q})](t-u)} \int_u^t{\rm d}w{\bf F}_{\rm d}(t)\cdot{\bf F}_{\rm d}(w)
		\\
		&\,\quad + {\cal O}(h^4,f^4) .
	\end{aligned}
\end{equation}
Substituting the explicit expression of the drive~\eqref{eq:theta}, and integrating over $u$ and $w$, we obtain
\begin{equation}
	\begin{aligned}
		{\cal J} - \frac{f^2}{\gamma} &= - \frac{(hf)^2}{d\gamma^3} \int \frac{{\rm d}{\bf q}}{(2\pi)^d} \frac{|{\bf q}|^4 |v({\bf q})|^2}{|{\bf q}|^4\big[D_0+D_{\rm G}K({\bf q})\big]^2 + \omega^2}
		\\
		&\quad\times \frac{D_0+D_{\rm G}K({\bf q})}{D_0K({\bf q})} + {\cal O}(h^4,f^4) .
	\end{aligned}
\end{equation}
The asymptotic results for the rate of work $\dot w = f^2/\gamma-{\cal J}$ in~(\ref{eq:small}-\ref{eq:large}) follow directly.


We now turn to deriving the diffusion coefficient $D$. It is defined in terms of the mean-squared displacement (MSD) $\langle\Delta{\bf r}_0^2(t)\rangle=\big\langle\big[\langle{\bf r}_0(t)\rangle - {\bf r}_0(t)\big]^2\big\rangle$ as $D=\underset{t\to\infty}{\lim}\langle\Delta{\bf r}_0^2(t)\rangle / 2 d t$. At leading order, the MSD reads $\langle\Delta{\bf r}_0^2(t)\rangle_0 = 2dD_0t$. To obtain the first order, we need to compute the following correlations
\begin{equation}
	\begin{aligned}
		\Big\langle\Delta{\bf r}^2_0(t)& \big[{\bf q}\cdot\bar{\bf r}_0(s)\big] {\rm e}^{{\rm i}{\bf q}\cdot[{\bf r}_0(s)-{\bf r}_0(u)]}\Big\rangle_0
		\\
		&= - 4 (D_0/\gamma)|{\bf q}|^2(s-u) \Theta(t-s)
		\\
		&\quad\times{\rm e}^{ - D_0|{\bf q}|^2(t-u) + \frac{{\rm i}{\bf q}}{\gamma}\cdot\int_u^t {\bf F}_{\rm d}(w) {\rm d}w},
		\\
		\Big\langle\Delta{\bf r}^2_0(t)& \big[\bar{\bf r}_0(u)\cdot\bar{\bf r}_0(s)\big] {\rm e}^{{\rm i}{\bf q}\cdot[{\bf r}_0(s)-{\bf r}_0(u)]}\Big\rangle_0
		\\
		&= (2/\gamma^2)\Theta(t-s) \big[2D_0|{\bf q}|^2(s-u)-1\big]
		\\
		&\quad\times{\rm e}^{ - D_0|{\bf q}|^2(t-u) + \frac{{\rm i}{\bf q}}{\gamma}\cdot\int_u^t {\bf F}_{\rm d}(w) {\rm d}w} ,
	\end{aligned}
\end{equation}
where we have used again that ${\cal A}_0$ is Gaussian in terms of $\bar{\bf r}_0$, yielding
\begin{equation}
	\begin{aligned}
		&\big\langle\Delta{\bf r}^2_0(t)\big\rangle - 2 d D_0 t
		\\
		&\,= \frac{2h^2}{\gamma^2} \int \frac{{\rm d}{\bf q}}{(2\pi)^d} \frac{|{\bf q}|^2 |v({\bf q})|^2}{K({\bf q})}
		\\
		&\,\quad\times \int_{-\infty}^t{\rm d}s \int_{-\infty}^s{\rm d}u \big\{ 2|{\bf q}|^2\big[D_0+D_{\rm G}K({\bf q})\big](s-u) - 1\big\}
		\\
		&\,\quad\times {\rm e}^{ -|{\bf q}|^2[D_0+D_{\rm G}K({\bf q})](s-u) + \frac{{\rm i}{\bf q}}{\gamma}\cdot\int_u^s{\bf F}_{\rm d}(w) {\rm d}w } + {\cal O}(h^4) .
	\end{aligned}
\end{equation}
Expanding at small $f$, we get
\begin{equation}
	\begin{aligned}
		&\big\langle\Delta{\bf r}^2_0(t)\big\rangle - 2 d D_{\rm eq} t
		\\
		&\,= - \frac{2h^2}{\gamma^4} \int \frac{{\rm d}{\bf q}}{(2\pi)^d} \frac{|{\bf q}|^4 |v({\bf q})|^2}{K({\bf q})}
		\\
		&\,\quad\times \int_{-\infty}^t{\rm d}s \int_{-\infty}^s{\rm d}u \big\{ 2|{\bf q}|^2\big[D_0+D_{\rm G}K({\bf q})\big](s-u) - 1\big\}
		\\
		&\,\quad\times {\rm e}^{ -|{\bf q}|^2[D_0+D_{\rm G}K({\bf q})](s-u)} \int_u^s {\rm d}w_1{\rm d}w_2 {\bf F}_{\rm d}(w_1)\cdot{\bf F}_{\rm d}(w_2)
		\\
		&\,\quad + {\cal O}(h^4,f^4) ,
	\end{aligned}
\end{equation}
where $D_{\rm eq}$ refers to the diffusion coefficient in the absence of driving force ($f=0$). From the explicit time integrations, we then obtain
\begin{equation}
	\begin{aligned}
		D - D_{\rm eq} &= \frac{(hf)^2}{d\gamma^4} \int \frac{{\rm d}{\bf q}}{(2\pi)^d} \frac{|{\bf q}|^2|v({\bf q})|^2}{K({\bf q})\big[D_0+D_{\rm G}K({\bf q})\big]}
		\\
		&\quad\times \frac{5|{\bf q}|^4\big[D_0+D_{\rm G}K({\bf q})\big]^2 + \omega^2}{ \big\{ |{\bf q}|^4\big[D_0+D_{\rm G}K({\bf q})\big]^2 + \omega^2 \big\}^2}
		\\
		&\quad + {\cal O}(h^4,f^4) .
	\end{aligned}
\end{equation}
Finally, we deduce the corresponding expressions in the asymptotic regimes in~(\ref{eq:small}-\ref{eq:large}).

Finally, considering a generalized periodic drive of the form ${\bf F}_{\rm d, gen}(t) = f \sum_n c_n\big[\sin(\omega_n t) \hat{\bf e}_x + \cos(\omega_n t)\hat{\bf e}_y \big]$, our perturbation can be extended straightforwardly to yield
\begin{equation}
	\begin{aligned}
		&{\cal J} - f^2/\gamma
		\\
		&\,= - \frac{(hf)^2}{d\gamma^3} \sum_n c_n^2 \int \frac{{\rm d}{\bf q}}{(2\pi)^d} \frac{|{\bf q}|^4 |v({\bf q})|^2}{|{\bf q}|^4\big[D_0+D_{\rm G}K({\bf q})\big]^2 + \omega_n^2}
		\\
		&\,\quad\times \frac{D_0+D_{\rm G}K({\bf q})}{D_0K({\bf q})} + {\cal O}(h^4,f^4) ,
		\\
	\end{aligned}
\end{equation}
and
\begin{equation}
	\begin{aligned}
		D - D_{\rm eq} &= \frac{(hf)^2}{d\gamma^4} \sum_n c_n^2 \int \frac{{\rm d}{\bf q}}{(2\pi)^d} \frac{|{\bf q}|^2|v({\bf q})|^2}{K({\bf q})\big[D_0+D_{\rm G}K({\bf q})\big]}
		\\
		&\quad\times \frac{5|{\bf q}|^4\big[D_0+D_{\rm G}K({\bf q})\big]^2 + \omega_n^2}{ \big\{ |{\bf q}|^4\big[D_0+D_{\rm G}K({\bf q})\big]^2 + \omega_n^2 \big\}^2}
		\\
		&\quad + {\cal O}(h^4,f^4) ,
	\end{aligned}
\end{equation}
showing that our results are indeed generic for a large class of driving forces.


% -------------------------------------------------------------------------------


\section{Numerical simulations}\label{app:simu}

In Sec.~\ref{sec:method}, numerical simulations of the dynamics~\eqref{eq:dyn} are performed using the LAMMPS simulation package in a two-dimensional box $10^2\sigma\times 10^2\sigma$, where $\sigma$ is the particle diameter, with periodic boundary conditions at average density $\rho_0=0.45$. The driving force ${\bf F}_{\rm d}$ only acts on ten percent of the particles with dynamics~\eqref{eq:theta}. The time step is $\delta t = 5\times 10^{-4} \tau_{\rm r}$. The system is first relaxed for $10^5$ time steps, and later equilibrated for $50\tau$. We evaluate average values over trajectories with duration of at least $150\tau$. Parameter values: $T=1$, $\gamma=10^2$, $v_0=1$, $\sigma=1$.


In Sec.~\ref{sec:bias}, a custom code of molecular dynamics, based on finite time difference, is used to perform the simulations in a two-dimensional box $10\sigma\times 10\sigma$ with periodic boundary conditions. The pair potential of $16$ particle pairs is biased in a system of $40$ particles in total. To sample the biased ensemble, we use the cloning algorithm described in Appendix A of~\cite{Nemoto2016}. The time interval for cloning is $\Delta t = 10 \delta t$ and the number of clones is $1600$. The time step is $\delta t = 10^{-4} \tau_{\rm r}$, the initial relaxation time is $10^4\Delta t$, and the total simulation time is $10^6 \Delta t$. Parameter values: $T=1$, $\gamma=1$, $v_0=4$ (Fig.~\ref{fig:outofperturbation}), $\sigma=1$.


% -------------------------------------------------------------------------------


\section{Equivalence of biased ensembles}\label{app:far}

In this Appendix, we demonstrate the equivalence between two dynamical biased ensembles: (a)~the equilibrium dynamics~\eqref{eq:dyn_eq} biased with $\sum_{i,j} \kappa_{ij}\int_0^t{\cal E}_{ij}(s){\rm d}s$, where ${\cal E}_{ij}$ is defined in~\eqref{eq:eps_bis}, and (b)~the first-order auxiliary dynamics, whose potential reads $V+2\sum_{i,j}\kappa_{ij}A({\bf r}_i-{\bf r}_j)$, biased with $\varepsilon'$ defined in~\eqref{eq:eps_pprime}. This amounts to showing that the path probabilities associated with (a) and (b) are similar. The corresponding dynamic actions, respectively denoted by ${\cal A}^{(\rm a)}(t)=\sum_k\int_0^t{\cal L}_k^{(\rm a)}(s){\rm d}s$ and ${\cal A}^{(\rm b)}(t)=\sum_k\int_0^t{\cal L}_k^{(\rm b)}(s){\rm d}s$, are given here in terms of the Lagragians ${\cal L}_k^{(\rm a)}$ and ${\cal L}_k^{(\rm b)}$ defined as
\begin{equation}
	\begin{aligned}
		{\cal L}_k^{(\rm a)} &= \frac{1}{4\gamma T} \big[\gamma\dot{\bf r}_k + \nabla_kV\big]^2 - \frac{1}{2\gamma} \nabla^2_kV
		\\
		&\quad - \frac{1}{\gamma T} \sum_{i,j}\kappa_{ij}\big[T \nabla_k - \nabla_k V\big] \cdot\nabla_k A({\bf r}_i-{\bf r}_j) ,
	\end{aligned}
\end{equation}
and
\begin{equation}
	\begin{aligned}
		{\cal L}_k^{(\rm b)}	&= \frac{1}{4\gamma T} \Big[\gamma\dot{\bf r}_k + \nabla_k V + 2\sum_{i,j}\kappa_{ij}\nabla_kA({\bf r}_i-{\bf r}_j)\Big]^2
		\\
		&\quad - \frac{1}{2\gamma} \nabla^2_k \Big[V + 2\sum_{i,j}\kappa_{ij} A({\bf r}_i-{\bf r}_j)\Big]
		\\
		&\quad - \frac{1}{\gamma T}\Big[\sum_{i,j}\kappa_{ij}\nabla_kA({\bf r}_i-{\bf r}_j)\Big]^2 .
	\end{aligned}
\end{equation}
Expanding ${\cal L}_k^{(\rm b)}$, it appears that ${\cal A}^{(\rm a)}$ and  ${\cal A}^{(\rm b)}$ are indeed equal, up to a boundary term proportional to $\sum_{i,j}\kappa_{ij}\big[A({\bf r}_i(t)-{\bf r}_j(t)) - A({\bf r}_i(0)-{\bf r}_j(0))\big]$ which can be neglected at large $t$.


% ===============================================================================


\bibliographystyle{apsrev4-1}
\bibliography{Driven_references.bib}

\end{document}


